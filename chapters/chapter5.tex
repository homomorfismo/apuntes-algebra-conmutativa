\documentclass[../main.tex]{subfiles}
\begin{document}
\section{Módulos de longitud finita y anillos artinianos}
\begin{definition}
Sea $M$ es un $A$-módulo. Decimos que $M$ es un $A$-módulo \textit{simple} si sus únicos submódulos son $\{0_M\}$ y $M.$
\end{definition}

\begin{remark} \textnormal{\textbf{(Caracterización de módulos simples)}} \label{obs_simple}
Si $M$ es un $A$-módulo simple, sea $m\in M$ no nulo, entonces el homomorfismo de $A$-módulos $A\overset{f}{\rightarrow} M$ dado por $1_A\mapsto m$, es suprayectivo ya que $\im f = \langle m \rangle$ que es submódulo de $M$ no nulo, luego $\im f = M$. Por el primer teorema de isomorfía,  $A/\ker{f}\cong M$. Además, el isomorfismo nos dice que $A/\ker f$ no tiene submódulos (ideales) no triviales, luego $A/\ker f$ es un cuerpo y por lo tanto $\ker f$ es un ideal maximal.
\end{remark}

\begin{definition}
Decimos que un $A$-módulo $M$ tiene \textit{longitud finita} si posee una cadena de submódulos
$$\{0_M\}=M_0\subsetneq M_1\cdots M_{n-1}\subsetneq M_n=M$$
tales que los cocientes $M_i/M_{i-1}$ son módulos simples. A dicha cadena la llamamos serie de composición. Si $M$ posee una serie de composición, llamamos \textit{longitud} de $M$, y lo denotamos por $l(M)$, al cardinal de la menor serie de composición. Si $M$ no posee una serie de composición decimos que tiene longitud infinita $l(M)=\infty.$ En particular, si $A$ visto como $A$-módulo tiene longitud finita se dice que es \emph{artiniano}.
\end{definition}

Esta noción mimetiza para módulos la idea de espacio vectorial de dimensión finita.

\begin{theorem}
Sea $M$ un $A$-módulo.
\begin{itemize}
    \item[i)] Si $N\subsetneq M$ es un submódulo, entonces $l(N)\leq l(M)$ y $l(N)<l(M)$ si $l(M)<\infty$.
    \item[ii)] Todas las series de composición de un módulo de longitud finita tienen el mismo cardinal, $l(M).$
    \item[iii)] (Transitividad) Si
    $$0\rightarrow M'\overset{f}{\rightarrow} M\overset{g}{\rightarrow}M''\rightarrow 0$$
    es una sucesión exacta de $A$-módulos, $l(M)<\infty$ si, y sólo si, $l(M'),l(M'')<\infty.$ Más aún, si se da cualquiera de esas condiciones se verifica
    $$l(M)=l(M')+l(M'').$$
    \item[iv)] Si $l(M)<\infty$, cada cadena de submódulos de $M$ se puede refinar a una serie de descomposición.
\end{itemize}
\end{theorem}
\begin{proof}

\textit{i)} Supongamos que $l(M)=n<\infty$ y sea $$\{0\}=M_0\subsetneq M_1\cdots M_{n-1}\subsetneq M_n=M$$ una serie de composición. Veamos que intersecando con $N$ y, si acaso borrando los términos redundantes, obtenemos una serie de composición de $N$. Si $0 \neq M_{i-1}\cap N\subsetneq M_{i}\cap N$, entonces la inyeccion $(M_i\cap N)/(M_{i-1}\cap N)\hookrightarrow M_i/M_{i-1}$ está bien definida y como $M_i/M_{i-1}$ es simple y $0 \neq (M_i\cap N)/(M_{i-1}\cap N)$ es un submódulo, han de ser iguales. Así, si $n'$ es el número de eslabones distintos que han quedado, tendremos $l(N)\le n'$ y, si $n'=n$, entonces $M=N.$

\textit{ii)} Lo probaremos por inducción sobre la longitud de alguna cadena de composición. Si  $M$ tiene una cadena de composición de un elemento, entonces el propio $M$ es simple y no tiene más submódulos que puedan alargar la cadena. Supongamos que es cierto que si un módulo $M$ tiene una serie de longitud $n-1$, todas las cadenas tienen esa longitud. Supongamos que $M$ tiene una serie de composición
$$\{0\}=M_0\subsetneq M_1\subsetneq\cdots\subsetneq M_n=M$$
El submódulo $M_{n-1}$ tiene la serie de composición $\{0\}=M_0\subsetneq M_1\subsetneq\cdots\subsetneq M_{n-1}$, luego por hipótesis de inducción $n-1=l(M_{n-1})<l(M_n)\le n$, y por tanto $l(M_n)=n.$

\textit{iii)} Si $l(M)< \infty$, por la inyectividad de $f$ podemos ver $M'$ como submódulo y entonces por \textit{i)} se tiene $l(M') \leq l(M)< \infty$. Para cualquier cociente cociente $M/N$, en particular para $N = \ker g$, si $\{0\}=M_0\subsetneq M_1\subsetneq\cdots\subsetneq M_n=M$ es una serie de composición, podemos tomar
$$
\{0_{M/N}\}\subsetneq \faktor{(M_1+N)}{N}\subsetneq\cdots\subsetneq \faktor{M_{n-1}+N}{N} \subsetneq M/N
$$
El homomorfismo de A-módulos $M_i/M_{i-1} \to (Mi + N)/(M_{i-1} + N)$ dado por $x + M_{i-1} \mapsto x + (M_{i-1} + N)$ es suprayectivo y como $M_i/M_{i-1}$ es simple o bien el núcleo es el nulo y obtenemos un isomorfismo, o bien es el total, y por tanto $(Mi + N) = (M_{i-1} + N)$.

Si $n'=l(M)<\infty$ y $n''=l(M'')<\infty$, entonces $M'$ tiene una serie de composición $\{M'_i\}_{i=1}^s$, y $M''\cong M/\ker g$ tiene una serie de composición $\{M_i/ \ker g\}_{i=1}^r$. Como los submódulos embotellados del cociente son submódulos embotellados de $M$ que contienen a $\ker g \cong M'$, podemos encajar la primera con la segunda tal que
$$M_0'\subsetneq M_1'\subsetneq\cdots\subsetneq\ker g\subsetneq M_1\subsetneq\cdots\subsetneq M_{n-1}\subsetneq M$$
es una serie de composición para $M.$ Este argumento prueba la aditividad.

\textit{iv)} Sea una cadena de submódulos $\{0\}=M_0\subsetneq M_1\subsetneq\cdots\subsetneq M_n=M$. Supongamos que $1 \leq i \leq n$ es tal que $M_i/M_{i-1}$ no es simple. Como $l(M) < \infty$, hemos visto en \textit{iii)} que $l(M/M_{i-1})< \infty$ luego por \textit{i)} tenemos $l(M_i/M_{i-1})< \infty$ también.
Tiene una serie de composición $0 = M_{i-1}/M_{i-1} \subsetneq \dots \subsetneq M_i/M_{i-1}$, que se traduce en una cadena de submódulos $M_{i-1} \subsetneq \dots \subsetneq M_i$, y cada cociente
$$M'_k/M_{k-1}' \cong (M'_k / M_{i-1}) / (M_{k-1}/M_{i-1}$$
y el segundo es simple porque es uno de los de la serie de composición del cociente. Podemos repetir esta operación con cada otro par de eslabones cuyo cociente no sea simple.

\end{proof}

\begin{corollary}\label{finit_gen_length}Sea $M$ un $A$-módulo.
\begin{itemize}
    \item[i)] Si $M$ es de longitud finita, entonces es noetheriano. En particular, todo anillo artiniano es noetheriano.
    \item[ii)] Si $M$ es un $A$-módulo finitamente generado y $A$ es un anillo artiniano, entonces $M$ es de longitud finita.
\end{itemize}
\end{corollary}
\begin{proof}
\textit{i)} Como todo submódulo es a su vez un módulo de longitud finita, es suficiente demostrar que $M$ es finitamente generado. Pues si algún submódulo fuese no finitamente. Sea $\{0\}=M_0\subsetneq M_1\subsetneq\cdots\subsetneq M_n=M$ una serie de composición. Como $M_i/M_{i-1}$ es simple, siguiendo la observación \ref{obs_simple}, existe $m_i\in M_i$ tal que  $M_i/M_{i-1}\cong \langle\overline{m_i}\rangle$.
En particular, $M_1/0 = M_1$ es simple así que $M_1 = \langle m_1 \rangle$. Probamos por inducción que $M=\langle m_1,\dots,m_n\rangle$.
Supongamos $M_{n-1}=\langle m_1,\dots,m_{n-1}\rangle$ y sea $x\in M$ cuya clase en $M/M_{n-1} = \langle m_n + M_{n-1}\rangle$ será tal que $x+M_{n-1}=\lambda m_n+ M_{n-1}$ con $\lambda\in A$. Por hipótesis, existen $\lambda_j\in A$ verificando $x-\lambda m_n=\sum_{j=1}^{n-1}\lambda_j m_j$ y despejando $x$ vemos que $m_1, \dots, m_n$ son generadores de $M$.

\textit{ii)} Primeramente, observamos que la suma directa $A^{(n)}$ es un $A$-módulo de longitud finita ya que tenemos la sucesión exacta $0\rightarrow A^{(n-1)}\overset{q_{n-1}}{\rightarrow}A^{(n)}\overset{p_1}{\rightarrow}A\rightarrow 0$ y podemos aplicar la proposición anterior recurrentemente. Supongamos ahora que $M$ está generado por $\{m_1,\dots,m_n\}$. Si $g:A^{(n)} \to M$ manda $e_i \mapsto m_i$ tenemos la sucesión exacta
$$0\rightarrow\ker g\rightarrow A^{(n)} \overset{g}{\rightarrow}M\rightarrow 0$$
y como $A^{(n)}$ es de longitud finita por la proposición anterior también lo es $M$.
\end{proof}

\begin{remark}\textnormal{\textbf{(Submódulos de $M$ como $A$-módulo y $A/\anul_A(M)$-módulo)}}\label{A/a-modulos}
Si $M$ es un $A$-módulo, $M$ tiene estructura natural de $A':=A/\anul_A(M)$-módulo y los $A$-submódulos de $M$ coinciden con los $A'$-submódulos de $M.$ Esto mismo ocurre como $A/\af$-módulo si $\af\subset\anul_A(M).$ Esta observación es muy útil y la aplicaremos en la siguiente demostración.
\end{remark}

\begin{proposition}
Sea $A$ un anillo. Las siguientes afirmaciones son equivalentes.
\begin{itemize}
    \item[i)] $A$ es artiniano.
    \item[ii)] $A$ es noetheriano y el $\langle 0\rangle$ es producto de ideales maximales.
    \item[iii)] $A$ es noetheriano y todo ideal primo es maximal.
\end{itemize}
En particular, si $A$ es un anillo noetheriano y $\af$ un ideal que es producto de ideales maximales, entonces $A/\af$ es un anillo artiniano.
\end{proposition}

\begin{proof} $i)\Rightarrow ii)$ $A$ es noetheriano por el último corolario. Sea $\langle 0_A\rangle\subsetneq \af_1\subsetneq\cdots\subsetneq\af_{n-1}\subsetneq\af_n=A$ una serie de composición.
Por la observación \ref{obs_simple} se tiene $\af_i/\af_{i-1}\cong A/\mathfrak{m}_i$ de manera que $\mathfrak{m}_i$ es anulador del cociente, ie. mete $\af_i$ en $\af_{i-1}$, como por ser maximales son coprimos tenemos que $\prod_{i=1}^n\mathfrak{m_i}= \bigcap_{i=1}^n \mathfrak{m_i}=\langle 0_A\rangle$.

$ii)\Leftrightarrow iii)$ Si $\prod_{i=1}^n\mathfrak{m_i}=\langle 0_A\rangle$ y $\p\in\Spec A$, entonces $\p\supset \prod_{i=1}^n\mathfrak{m_i}=\bigcap_{i=1}^n \mathfrak{m_i}$ y por ser $\p$ primo $\p\supset\mathfrak{m_j}$ para algún $j$ y por maximalidad $\p=\mathfrak{m_i}$.
Recíprocamente, por ser $A$ noetheriano, $\langle 0\rangle$ es intersección de primarios cuyos primos asociados son maximales. Así, $\sqrt{\langle 0\rangle}=\cap_{i=1}^l\mathfrak{m_i} = \prod_{i=1}^l\mathfrak{m_i}$ con $\mathfrak{m_i}$. Por ser los ideales finitamente generados, existe $\nu\in\N$ tal que
$$\langle 0\rangle = \left(\prod_{i=1}^l\mathfrak{m_i}\right)^\nu=\prod_{i=1}^l(\mathfrak{m_i)^\nu}.$$

($ii)\Rightarrow i)$) Supongamos que $\langle 0\rangle=\mathfrak{m}_1\cdots\mathfrak{m}_r = mathfrak{m}_1\cap\cdots\cap \mathfrak{m}_r$ posiblemente repetidos, que es una descomposición primaria del $\langle 0\rangle$ que podemos suponer irrendundante. Veamos que $A$ es artiniano por inducción sobre el menor número de maximales necesarios en esa factorización. Si éste es 1, entonces el anillo $A$ es un $A$-modulo simple. Supongamos que si $0$ se descompone en un producto de $r-1$ maximales entonces el anillo es artiniano y, sin pérdida de generalidad, veamos qué pasa si $\langle 0\rangle=\mathfrak{m}_1\cdots\mathfrak{m}_r$ pero $0 \neq \mathfrak{m} := \mathfrak{m}_1\cdots\mathfrak{m}_{r-1}$.
Tenemos la sucesión exacta de $A$-módulos:
$0\rightarrow \mathfrak{m}\rightarrow A\rightarrow A/\mathfrak{m} \rightarrow 0$
y podemos ver que la longitud de $A$ es finita viendo que la de los extremos lo es.

En primer lugar, $\mathfrak{m}$ es un $A/\mathfrak{m}_1$-espacio vectorial finitamente generado, luego de dimensión finita, y por tanto de longitud finita. Por la observación \ref{A/a-modulos}, también es un $A$-módulo de longitud finita. Por otra parte, nuevamente aplicando la observación, basta ver que $A/\mathfrak{m}$ tiene longitud finita como $A/\mathfrak{m}$-módulo.
El ideal $0$ en $A/\mathfrak{m}$ es precisamente $ \mathfrak{m}/ \mathfrak{m} = \overline{\mathfrak{m}_1}^{e_1-1}\cdots\overline{\mathfrak{m}_{r-1}}$ donde $\overline{\mathfrak{m}_i}:=\mathfrak{m}_i+\mathfrak{m}_1^{e_1-1}\cdots\mathfrak{m}_r^{e_r}$.
Este es un producto de $r-1$ maximales y por tanto $A/\mathfrak{m}$ es un anillo artiniano, lo que concluye la demostración.
\end{proof}

\begin{proposition}Sea $A$ un anillo noetheriano y $M$ un $A$-módulo. Las siguientes afirmaciones son equivalentes.
\begin{itemize}
    \item[i)] $l(M)<\infty$.
    \item[ii)] $M$ es noetheriano y los ideales de $\Ass_A(M)$ son maximales.
    \item[iii)] $M$ es noetheriano y $\sqrt{\anul_A(M)}$ es producto de ideales maximales
\end{itemize}
\end{proposition}
\begin{proof}
$i)\Rightarrow ii)$ La noetherianidad ya está probada. Sea $\{0\}=M_0\subsetneq M_1\subsetneq\cdots\subsetneq M_n=M$ una sere de composición y probamos la afirmación por inducción sobre $l(M)$. Si $l(M) = 1$, entonces $M$ es simple y $M\cong A/\mathfrak{m}$ con $\mathfrak{m}$ maximal, y como el cociente es un cuerpo los únicos elementos de $A$ que son anuladores son los de $\mathfrak{m}$, luego $\Ass_A(M) = \{\mathfrak{m}\}$. Supongamos cierto el resultado para todo $r < l(M)$ y observamos que la sucesión
$$0\rightarrow M_{n-1}\rightarrow M\rightarrow M/M_{n-1}\rightarrow 0$$
es exacta. Por un lado, $M/M_{n-1}\cong A/\mathfrak{m'}$ con $\mathfrak{m'}$ maximal, y como antes $\Ass_A(M/M_{n-1}) = \{\mathfrak{m}'\}$. Por otro lado, como $l(M_{n-1})<l(M)$, $\Ass_A(M_{n-1})$ son ideales maximales por hipótesis de inducción. Finalmente, por el lema \ref{contenidos_assa} sabemos que
$$\Ass_A(M)\subset \left ( \Ass_A(M_{n-1})\cup(\Ass_A(M/M_{n-1}) \right)=\Ass_A(M_{n-1})\cup\{\mathfrak{m}\} $$.
lo que concluye la prueba.

$ii)\Rightarrow iii)$ Los ideales minimales de $\anul_A(M)$, están $\Ass_A(M)$, luego son un número finito. Así, $\sqrt{\anul_A(M)}=\cap_{\p\in\Ass_A(M)}\p=\mathfrak{m_1}\cap\cdots\cap\mathfrak{m_r}=\mathfrak{m_1}\cdots\mathfrak{m_r}$

$iii)\Rightarrow i)$ Como los ideales son maximales tenemos
$$\sqrt{\anul_A(M)}=\mathfrak{m_1}\cap\cdots\cap\mathfrak{m_r}=\mathfrak{m_1}\cdots\mathfrak{m_r}$$
Por ser todos ellos finitamente generados, existe $\nu\in\N$ tal que $(\mathfrak{m_1}\cdots\mathfrak{m_r})^\nu\subset\anul_A(M).$ Ahora, aplicando de nuevo la observación anterior, basta ver que $M$ es de longitud finita como $A':=A/(\mathfrak{m_1}^\nu\cdots\mathfrak{m_r}^\nu)$-módulo. $M$ es finitamente generado y $A'$ es de longitud finita por la proposición anterior, luego $M$ es un $A'$-módulo de logitud finita por el corolario \ref{finit_gen_length}.
\end{proof}

\begin{remark}
Sea $A$ un anillo noetheriano y $\q$ un ideal $\p$-primario con $\p$ ideal primo. Consideremos la localización $A_\p$, cuyo único ideal maximal es $\p^e.$ Sabemos que $\q^e$ es $\p^e$-primario. Es por esto que $\sqrt{\q^e}=\p^e$ y, por ser finitamente generado, existe $\nu\in\N$ tal que $(\p^e)^\nu\subset \q^e.$ Así, el anillo $A_\p/(\p^e)^\nu$ es un anillo artiniano y también un $A_\p$-módulo artiniano. Por esto, el cociente $A_\p/\q^e$ también lo será.
\end{remark}

\begin{definition}
En las condiciones de la observación anterior, llamamos longitud de $\q$ a la longitud de $A_\p/\q^e.$
\end{definition}
\end{document}
