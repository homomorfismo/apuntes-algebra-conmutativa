\documentclass[../main.tex]{subfiles}
\begin{document}
\section{Anillos y módulos noetherianos}
\begin{proposition}
Sea $A$ un anillo y $M$ un $A$-módulo. Las siguientes afirmaciones son equivalentes.\begin{enumerate}
    \item Todo conjunto no vacío de submódulos de $M$ tiene un elemento maximal respecto del contenido.
    \item Toda sucesión ascendente de submódulos de $M$ $M_1\subset M_2\subset\dots\subset M_n\subset\dots$ es estacionaria, es decir, existe un $k\in\N$ tal que $M_k=M_{k+l}$ para todo $l\in\N$.
    \item Todo submódulo de $M$ es finitamente generado
\end{enumerate}
Si $M$ verifica cualquiera de estas condiciones equivalentes se dice que es un $A$-módulo \textit{noetheriano}. Un anillo $A$ se dice que es un \textit{anillo noetheriano} si, visto como un $A$-módulo, es noetheriano.
\end{proposition}
\begin{proof} Vamos probando cada una de las implicaciones.

($1\Rightarrow 2$) Sea $M_k$ el elemento maximal del conjunto $\{M_n:n\in\N\}$ formado por cada uno de los submódulos de la cadena. Necesariamente, para cada $l\in\N$, $M_k=M_{k+l}$, por ser la cadena ascendente y para no contradecir la maximalidad de $M_k$.

($2\Rightarrow 3$) Sea $N\subset M$ un submódulo arbitrario. Supongamos que $N$ no es finitamente generado. Entonces, $N\neq\{0\}$. Sea $f_1\in N\setminus\{0\}$. Como $N$ no es finitamente generado, $\langle f_1\rangle\varsubsetneq N$. Sea $f_2\in N\setminus\langle f_1\rangle$. Como $N$ no es finitamente generado, $\langle f_1,f_2\rangle\varsubsetneq M$. Inductivamente, generamos una sucesión $\langle f_1\rangle\varsubsetneq\langle f_1,f_2\rangle\varsubsetneq\dots\varsubsetneq\langle f_1,\dots,f_n\rangle\varsubsetneq\dots$ de $A$-módulos no estacionaria, contradiciendo la hipótesis.

($3\Rightarrow 2$) Sea $\Sigma$ un conjunto no vacío de submódulos de $M$, ordenados por la inclusión. Tomemos $\{N_i:i\in I\}$ una cadena de $\Sigma$. Definiendo $N^{\ast}=\bigcup_{i\in I}N_i\subset M$. Por hipótesis, $N^{\ast}$ es finitamente generado. Sean $y_1,\dots,y_l$ sus generadores. Supongamos que cada $y_j\in N_{i_j}$. Sea $N_k=\operatorname{max}\{N_{i_1},\dots,N_{i_l}\}$. Como $\{N_i\}$ es una cadena, necesariamente se tiene $N^{\ast}=N_k$. 

Hemos visto que toda cadena de $\Sigma$ tiene máximo. Por el Lema de Zorn, $\Sigma$ tiene elemento maximal.
\end{proof}
\begin{proposition}
Dada una sucesión corta y exacta de homomorfismos de $A$-módulos $$0\longrightarrow M'\overset{f}{\longrightarrow}M\overset{g}{\longrightarrow}M''\longrightarrow 0$$ $M$ es noetheriano si y solo si $M'$ y $M''$ son noetherianos.
\end{proposition}
\begin{proof} Utilizamos las diferentes caracterizaciones de los $A$-módulos noetherianos descritas en la proposición anterior.

($\Longrightarrow$) Una cadena ascendente de submódulos de $M'$ lo es también de $M$ por ser $f$ inyectiva. Usando $(2)$, dicha cadena es estacionaria y por tanto $M'$ es noetheriano.

Como $g$ es sobreyectiva, $M''\cong\faktor{M}{\operatorname{ker} g}$. Por el Teorema de la correspondencia, los submódulos de $\faktor{M}{\operatorname{ker} g}$ son de la forma $\faktor{T}{\operatorname{ker} g}$ con $T\supset \operatorname{ker} g$ un submódulo de M. Como M es noetheriano, por $3$, $T$ es finitamente generado y por tanto, $\faktor{T}{\operatorname{ker} g}$ también y $M''$ es noetheriano.

($\Longleftarrow$) Sea $N\subset M$ un submódulo. $f^{-1}(N)$ es un submódulo de $M'$, luego es finitamente generado. Sean $x_1,\dots,x_r$ sus generadores. Usando que $M''\cong\faktor{M}{\operatorname{ker} g}$, $\faktor{(N+\operatorname{ker} g)}{\operatorname{ker} g}$ es un submódulo de $M''$ y entonces es finitamente generado. Sean $\overline{y_1},\dots,\overline{y_k}$ sus generadores. Veamos que $\langle f(x_1),\dots,f(x_r),y_1,\dots,y_k\rangle$ generan $N$.

Dado $z\in N$, existen $\lambda_i\in A$, $i=1,\dots,k$ tal que $$\overline{z}=\sum_{i=1}^k\lambda_i\overline{y_i}$$ Se tiene que $w=z-\sum_{i=1}^k\lambda_iy_i\in\operatorname{ker} g=\operatorname{Im} f$, luego existe $u\in M'$ tal que $f(u)=w$. Existen $\mu_j\in A$, $j=1,\dots,r$ tal que $u=\sum_{j=1}^r\mu_jx_j$. Aplicando $f$, se cumple $$w=\sum_{j=1}^r\mu_jf(x_j)$$ Con todo se tiene que $$z=\sum_{j=1}^r\mu_jf(x_j)+\sum_{i=1}^k\lambda_iy_i$$
\end{proof}
\begin{remark}
De este resultado se siguen las siguientes consecuencias.\begin{enumerate}
    \item Los cocientes y submodulos de los $A$-módulos noetherianos son noetherianos.
    \item Si $M_1,\dots,M_r$ son $A$-módulos noetherianos, $\bigoplus_{i=1}^rM_i$ es noetheriano.
    \item Sea $M$ un $A$-módulo finitamente generado, donde $A$ es un anillo noetheriano. Entonces $M$ es un $A$-módulo noetheriano. En efecto, si $r$ es un número de generadores de $M$, por ser $A$ noetheriano, $A^{(r)}$ es noetheriano tambíen y se genera la sucesión exacta $A^{(r)}\rightarrow M\rightarrow 0$.
\end{enumerate}
\end{remark}
El Teorema de la base de Hilbert del primer capítulo nos garantiza que si $A$ es un anillo noetheriano, $A[X]$ es también noetheriano. Inductiavente se ve que $A[X_1,\dots,X_n]$ es también noetheriano. 

A su vez, si recordamos la definición de $A$-álgebra finitamente generada del primer capítulo, se tiene que existe un homomorfismo suprayectivo de $A[X_1,\dots,X_n]$ en $B$, donde $B$ es la $A$-álgebra y $n$ es el número de generadores de $B$ como $A$-álgebra. Entonces, si $A$ es un anillo noetheriano, cualquier $A$-álgebra finitamente generada es un anillo noetheriano.

\section{Asociados primos y descomposición primaria}
Sea $A$ un anillo noetheriano y $\af$ un ideal. El objetivo de esta sección es descomponer $\af$ como intersección finita $\af=\bigcap \q_i$, con cada $\q_i$ primario asociado a un ideal primo $\p_i$.

Buscamos que tal descomposición sea \textit{irredundante}. Esto es que para cada $i$ se verifique $\q_i\nsupseteq \bigcap_{j\neq i}\q_j$.

En $\faktor{A}{\af}$, esto es equivalente a hacer la descomposición sobre el $0$ de $\faktor{A}{\af}$.

\begin{remark}
\begin{enumerate}
    \item Tal descomposición no tiene por qué ser única. Si tomamos por ejemplo en el anillo $A=K[X,Y]$ el ideal $\af=\langle x^2,xy\rangle$, se cumple $\af=\langle x\rangle\cap\langle x^2,y=\langle x\rangle\cap\langle x^2,xy,y^2\rangle$.
    \item  Veremos más adelante que los asociados primos sí son únicos. En el caso anterior serían $\langle x\rangle, \langle x,y\rangle$.
\end{enumerate}
\end{remark}
\end{document}