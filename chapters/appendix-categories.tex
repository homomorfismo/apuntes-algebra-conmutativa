\documentclass[../main.tex]{subfiles}
\begin{document}
Una categoría $\zeta$ viene dada por:\begin{itemize}
    \item La \textit{clase} de sus objetos $Obj(\zeta)$.
    \item Para cada par de objetos $A$, $B\in Obj(\zeta)$ un conjunto llamado $Hom_{\zeta}(A,B)$, las ``flechas" de $A$ en $B$.
    \item Para cada $A$, $B$, $C\in Obj(\zeta)$ una aplicación
    $$\begin{array}{rcl}
    Hom_{\zeta}(A,B)\times Hom_{\zeta}(B,C)&\longrightarrow&Hom_{\zeta}(A,C)\\
    (f,g)&\longmapsto&g\circ f
    \end{array}$$
    siendo dichas aplicaciones asociativas.
\end{itemize}
\begin{definition} Un funtor covariante entre dos categorías $\zeta$ y $\zeta '$ es una aplicación entre sus objetos
 $$\begin{array}{rcl}
    F: Obj(\zeta)&\longrightarrow&Obj(\zeta ')\\
    A&\longmapsto&F(A)
    \end{array}$$
y para cada $A,B\in Obj(\zeta)$ una aplicación
$$\begin{array}{rcl}
    F: Hom_{\zeta}(A,B)&\longrightarrow&Hom_{\zeta '}(F(A),F(B))\\
    f&\longmapsto&F(f)
    \end{array}$$
tal que se verifica \begin{itemize}
    \item [1)] Para cada $C\in Obj(\zeta)$ y para cada $f\in Hom_{\zeta}(A,B)$ y $g\in Hom_{\zeta '}(B,C)$, $F(g\circ f)=F(g)\circ F(f)$
    \item [2)] Para cada cada $A\in Obj(\zeta)$, $F(1_A)=1_{F(A)}$
\end{itemize}
\end{definition}
Nótese que hemos empleado la misma notación, $F$, para definir dos funciones en principio distintas, pero se permite este abuso de notación ya que se puede distinguir muy fácilmente sobre qué conjunto está actuando la $F$ en cada momento.
\begin{example}
\textit{1)} Sea $\zeta_{TOP}$ la categoría de los espacios topológicos y $\zeta_{SET}$ la categoría de los conjuntos. Definimos en funtor \textit{olvido} como
$$
\begin{array}{rcl}
    F: \Obj(\zeta_{TOP})&\longrightarrow&\Obj(\zeta_{SET})\\
    X&\longmapsto&X
\end{array}
$$
\textit{2)} Sea $G_T$ la categoría de grupos, podemos definir un funtor$$F:Obj(\zeta_{SET})\longrightarrow Obj(G_T)$$ asociando a cada conjunto $X$ el grupo libre generado por $X$, es decir, el conjunto de palabras generado por $X$.

\textit{3)} Sea $Ann$ la categoría de anillos conmutativos unitarios. Dado $A\in Obj(Ann)$, consideramos $Mod_A$ la categoría de $A$-módulos. Dado $M\in Obj(Mod_A)$, definimos el funtor covariante
$$\begin{array}{rcl}
    Hom_A(M,\_):Mod_A&\longrightarrow&Mod_A\\
    N&\longmapsto&Hom_A(M,N)
    \end{array}$$
A su vez, dados $N_1, N_2$ $A$-módulos y $f:N_1\rightarrow N_2$ homomorfismo, podemos definir
$$\begin{array}{rcl}
    f_{\ast}:Hom_A(M,N_1)&\longrightarrow&Hom_A(M,N_2)\\
    \varphi&\longmapsto&f\circ\varphi
    \end{array}$$
Si tenemos la secuencia de homomorfismo de $A$-módulos $$N_1\overset{f}{\longrightarrow}N_2\overset{g}{\longrightarrow}N_3$$
se tiene la siguiente secuencia $$Hom_A(M,N_1)\overset{f_{\ast}}{\longrightarrow}Hom_A(M,N_2)\overset{g_{\ast}}{\longrightarrow}Hom_A(M,N_3)$$ que verifica $(g\circ f)_{\ast}=g_{\ast}\circ f_{\ast}$
\end{example}
\begin{definition} Un funtor contravariante entre dos categorías $\zeta$ y $\zeta '$ consiste en la aplicación
 $$\begin{array}{rcl}
    F: Obj(\zeta)&\longrightarrow&Obj(\zeta ')\\
    A&\longmapsto&F(A)
    \end{array}$$
y para cada $A,B\in Obj(\zeta)$ una aplicación
$$\begin{array}{rcl}
    F: Hom_{\zeta}(A,B)&\longrightarrow&Hom_{\zeta '}(F(B),F(A))\\
    f&\longmapsto&F(f)
    \end{array}$$
tal que se verifica \begin{itemize}
    \item [1)] Para cada $C\in Obj(\zeta)$ y para cada $f\in Hom_{\zeta}(A,B)$ y $g\in Hom_{\zeta '}(B,C)$, $F(g\circ f)=F(f)\circ F(g)$
    \item [2)] Para cada cada $A\in Obj(\zeta)$, $F(1_A)=1_{F(A)}$
\end{itemize}
\end{definition}
Al igual que antes, hacemos un abuso de notación al usar $F$ para denotar funciones distintas.
\begin{example}
Consideremos $\zeta_{TOP}$ la categoría de espacios topológicos con aplicaciones continuas. Tomamos
$$\begin{array}{rcl}
    F: Obj(\zeta_{TOP})&\longrightarrow&Obj(Ann)\\
    (X,T)&\longmapsto&Cont(X,\mathbb{R})
    \end{array}$$
donde $Cont(X,\mathbb{R})$ es el conjunto de las aplicaciones continuas de $X$ a $\mathbb{R}$. Este conjunto es un anillo conmutativo y unitario con las operaciones $(f+g)(x)=f(x)+g(x)$ y $(f\cdot g)(x)=f(x)\cdot g(x)$

Dado $f:X\rightarrow Y$ continua, le asociamos el funtor contravariante
$$\begin{array}{rcl}
    Cont(Y,\mathbb{R})&\longrightarrow&Cont(X,\mathbb{R})\\
    \varphi&\longmapsto&\varphi\circ f
    \end{array}$$
\end{example}
\begin{definition} Sea $\zeta$ una categoría.\begin{itemize}
    \item [1)]Sea $O\in Obj(\zeta)$ tal que para cada $A\in Obj(\zeta)$, $Hom_{\zeta}(O,A)$ es un único elemento. Entonces a $O$ se le llama objeto inicial de una categoría
    \item [2)] Sea $O\in Obj(\zeta)$ tal que para cada $A\in Obj(\zeta)$, $Hom_{\zeta}(A,O)$ es un único elemento. Entonces a $O$ se le llama objeto final de una categoría
\end{itemize}
\end{definition}
\begin{example}
\textit{1)} $\varnothing$ es un objeto inicial.

\textit{2)} \{x\} es un objeto final

\textit{3)} Dado $A\in \Obj(\Ann)$, $\Mod_A$ tiene a $\{0\}$ como objeto inicial y final
\end{example}
\begin{definition} Dadas una categoría $\zeta$, $A, A', B, B'\in Obj(\zeta)$ y $u\in \Hom_{\zeta}(A,B)$, \begin{itemize}
    \item [1)] Decimos que $u$ es un monomorfismo si $u\circ f=i\circ g$ implica que $f=g$, donde $f$ y $g$ pertenecen a $\Hom_{\zeta}(A',A)$
    \item[2)] Decimos que $u$ es un epimorfismo si $f\circ u=g\circ u$ implica que $f=g$, donde $f$ y $g$ pertenecen a $\Hom_{\zeta}(B,B')$
\end{itemize}
\end{definition}
\begin{remark}
\textit{1)} Si tomamos las categorías de anillos y módulos, los conceptos de monomorfismo e inyectividad son equivalentes.

\textit{2)} En la categoría de módulos, el concepto de epimorfismo es equivalente al de homomorfismo suprayectivo. En la categoría de anillos, homomorfismo suprayectivo sí implica epimorfismo, pero no se tiene la otra implicación. En efecto, $$\mathbb{Z}\hookrightarrow \mathbb{Q}\overset{f,g}{\longrightarrow}C$$con $C$ anillo verifica las condiciones de epimorfismo $f\restriction_{\mathbb{Z}}= g\restriction_{\mathbb{Z}}$ implica $f=g$, pero la inclusión de $\mathbb{Z}$ sobre $\mathbb{Q}$ no es sobreyectiva
\end{remark}
\end{document}
