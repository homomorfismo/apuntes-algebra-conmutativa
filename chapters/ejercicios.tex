\documentclass[../main.tex]{subfiles}
\begin{document}

\section{Hoja 1}

\paragraph{Ejercicio 1} Sea $u\in A$ una unidad y $x\in A$ un elemento nilpotente. Demostrar que $u+x$ es una unidad.

Comenzamos probando que si $x\in \mathfrak N_A$, entonces $1+x\in \mathcal U(A)$. Existe $n>0$ tal que $x^n=0$, y entonces observamos que $(1+x)x^{n-1}=x^{n-1}$. Así:
\begin{multline}
(1+x^{n-1})(1+x) = 1+2x^{n-1} =1+2x^{n-1}(1+x)\\
=(1+x^{n-1})(1+x)-2x^{n-1}(1+x)=1\\
=(1+x^{n-1}-2x^{n-1})(1+x)=1\\
=1-x^{n-1})(1+x)=1
\end{multline}

Por otra parte, si $u\in \mathcal U(A)$, existe $v\in A$ tal que $uv=1$. Además, por ser $\mathfrak N_A$ un ideal, $vx \in \mathfrak N_A$ con mismo índice de nilpotencia, y podemos aplicar lo anterior
$$
(1-(vx)^{n-1})(1+vx)=1
$$
Ahora podemos escribir $1+vx = v(u+x)$ y por tanto la anterior identidad queda escrita como
$$
[v(1-(vx)^{n-1})](u+x)=1
$$


\paragraph{Ejercicio 2} Sea $A,A_1,A_2$ anillos y supongamos que $A\cong A_1\times A_2$.
\begin{enumerate}[label=(\roman*)]
    \item Sea $\mathfrak a \subset A$ un ideal. Demostrar que $\mathfrak a \cong \mathfrak a' \times \mathfrak a''$ para ciertos ideales $\mathfrak a' \subset A_1$ y $\mathfrak a'' \subset A_2$.
    \item Sea $\mathfrak p \subset A$ un ideal primo. Demostrar que $\mathfrak p \cong \mathfrak  p' \times A_2$ o bien $\mathfrak p \cong A_1 \mathfrak p''$ para ciertos ideales primos $\mathfrak p' \subset A_1$ y $\mathfrak p'' \subset A_2$.
\end{enumerate}

(i) En general, si $\phi:A\to B$ es un isomorfismo, y $\mathfrak a \subset A$ un ideal, entonces $\phi(\mathfrak a)$ es un ideal de $B$:

- Para todo $\phi(x),\phi(y)\in \phi(\mathfrak a)$ tenemos que $\phi(x)+\phi(y) = \phi(\underset{\in \mathfrak a}{x+y}) \in \phi(\mathfrak a)$.
- Para todo $\phi(x)\in \phi(\mathfrak a),z\in B$ existe $w\in A$ tal que $\phi(w) = z$, y entonces $z\phi(x) = \phi(\underset{\in \mathfrak a}{wx}) \in \phi(\mathfrak a)$.

Y todo ideal del producto $\mathfrak b \subset A_1\times A_2$, es un producto de ideales $\mathfrak b_1 \times \mathfrak b_2$. Efectivamente, sea
$$
\mathfrak b_1 = \{x\in A_1: \; \exists y\in A_2//(x,y)\in \mathfrak b\}
$$
y veamos que es un ideal:

- Para todo $x, x' \in \mathfrak b_1$ existen $y,y' \in A_2$ tales que $(x,y),(x',y')\in \mathfrak b$ y por ser un ideal tenemos $\mathfrak b \ni (x,y)+(x',y') = (x+x',y+y')$ y por tanto $x+x' \in \mathfrak b_1$.
- Para todo $x\in \mathfrak b_1$ y todo $z\in A_1$ existe $y\in A_2$ tal que $(x,y)\in \mathfrak b$, y además $(z,0)\in A_1\times A_2$, y por ser un ideal se tiene $\mathfrak b\ni (x,y)(z,0) = (xz,0)$ con lo que $xz \in \mathfrak b_1$.

Con esto queda probado que todo $\mathfrak a \subset A$ es isomorfo a un producto de ideales.

(ii) En general, si $\phi:A\to B$ es un isomorfismo, y $\mathfrak p \subset A$ un ideal primo, entonces $\phi(\mathfrak p)$ es un ideal primo de $B$:

- Sean $x',y' \in B$ tales que $x' = \phi(x), y' = \phi(y)\in \phi(\mathfrak p)$, entonces $\phi(\mathfrak p) \ni x'y' = \phi(x)\phi(y) = \phi(xy) $ por tanto $xy\in \mathfrak p$ y como es un ideal primo, $x \in \mathfrak p$ o $y\in \mathfrak p$ $\iff$ $x' \in \phi(\mathfrak p)$ o $y' \in \phi(\mathfrak p)$.

Si $\mathfrak p \subset A_1\times A_2$ es un ideal primo, entonces sabemos de a)  que $\mathfrak p = \mathfrak a_1\times \mathfrak a_2$ producto de ideales. Veamos que o bien $\mathfrak p = \mathfrak p_1\times A_2$ con $\mathfrak p_1$ primo, o bien $\mathfrak p = A_1\times \mathfrak p_2$ con $\mathfrak p_2$ primo. Supongamos $\mathfrak p_1 \neq A_1$:

- Para todo $x,y \in A_1$ tales que $xy\in \mathfrak p_1$  existe $z\in A_2$ tal que $(xy,z)\in \mathfrak p$. Entonces se tiene $\mathfrak p \ni (xy,z) = (x,z)(y,1)$ y por lo tanto $(x,z)\in \mathfrak p$ o bien $(y,1)\in \mathfrak p$ lo que implica que $x\in \mathfrak p_1$ o $y \in \mathfrak p_1$. Por tanto $\mathfrak p_1$ es un ideal primo.
- Más aún, dado $x\in \mathfrak p_1$, obviamente se cumple $1\cdot x \in \mathfrak p_1$. Siguiendo lo de arriba, $(1,z)(x,1) \in \mathfrak p$, y como $\mathfrak p_1\neq A_1$ no puede ser que $(1,z)\in \mathfrak p$, luego necesariamente $(x,1)\in \mathfrak p$ y por lo tanto $1\in\mathfrak p_2$ y así $\mathfrak p_2 = A_2$.



\paragraph{Ejercicio 3} Sea $\mathfrak a \subset A$ un ideal. Demostrar que:
\[\sqrt{\mathfrak a} = \bigcap_{\substack{\mathfrak p\in \operatorname{Spec}(A)\\ \mathfrak a \subset \mathfrak p}} \mathfrak p\]

Utilizando la caracterización que conocemos del nilradical de un anillo aplicado al cociente, y teniendo en cuenta que la biyección del teorema de la correspondencia conserva la primalidad, tenemos que:
\begin{multline}
x\in \sqrt {\mathfrak a} \iff x+\mathfrak a \in \mathfrak N_{A/\mathfrak a} = \bigcap_{\bar {\mathfrak p}\in \operatorname{Spec}(A/\mathfrak a)}\bar {\mathfrak p} \iff\\ \forall \bar {\mathfrak p} \in \operatorname{Spec}(A/\mathfrak a), \, x+\mathfrak a \in  \bar {\mathfrak p}  \iff\\ \forall \mathfrak p\in\operatorname{Spec}(A),\;  x\in \mathfrak p
\end{multline}


\paragraph{Ejercicio 4} Sea $A$ un anillo y $f=a_nX^n + \ldots + a_1 X+ a_0 \in A[X]$. Demostrar que $f$ es una unidad en $A[X]$ si y solo si $a_0$ es unidad y todos los $a_i$ son nilpotentes.

$\Leftarrow)$ Sabemos que $\mathfrak N_A$ es un ideal, así que $\sum_{j=1}^n a_j X^j\in \mathfrak N_A$, y como $a_0\in \mathcal U(A)$, en virtud del ejercicio 1 se tiene que $\sum_{j=1}^n a_j X^j+a_0 = f \in \mathcal U(A)$.

$\Rightarrow)$ Como $f$ es una unidad, existe $g =\sum_{j=1}^m b_j X^j \in A[X]$ tal que $fg = 1$. En primer lugar, esto implica que $a_0b_0 = 1$ luego $a_0\in \mathcal U(A)$.

FALTA LA SEGUNDA PARTE

\paragraph{Ejercicio 5} \textit{Sea $A$ un DIP. Si $\mathfrak a $ es un ideal propio, demostrar que son equivalentes
\begin{enumerate}[label=\alph*)]
    \item $\mathfrak a$ es un ideal primo,
    \item $\mathfrak a$ es un ideal maximal,
    \item existe $f\in A$ irreducible tal que $\mathfrak a = \gen{f}$.
\end{enumerate}
Si $a,b \in A \setminus \{0\}$ no son unidades, y $d,m\in A$ tales que $\gen{a} + \gen{b} = \gen{d}$, $\gen{a}\cap \gen{b} = \gen{m}$, demostrar que $d = \gcd(a,b)$ y $m = \operatorname{lcm}(a,b)$.}

$a)\iff b)$ La implicación $\Leftarrow$ se tiene siempre. Sea $\mathfrak a = aA$ un ideal primo, y supongamos que existe $\mathfrak b = bA$ tal que $\mathfrak a\subsetneq \mathfrak b$. Existe $x\in A$ tal que $bx = a\in \mathfrak a$ primo, luego $b\in \mathfrak a$ o $x\in \mathfrak a$. No puede ser que $b\in \mathfrak a$ porque en tal caso existiría un $z\in A$ tal que $az = b$ y entonces para todo $t\in  A$ se tendría que $bt = a(zt) \in aA = \mathfrak a$ y por tanto $\mathfrak b \subseteq \mathfrak a$, en contra de nuestra hipótesis. Por tanto $x \in \mathfrak a$, y existe $w\in A$ tal que $x = aw$, entonces $a(bw) = a$ y por tanto $1=bw   \in \mathfrak b$, con lo que $\mathfrak b = A$. Así $\mathfrak a$ es maximal.

$b)\iff c)$ Sea $\mathfrak a =aA$ un ideal, y supongamos que $a$ se puede expresar como $a=uv$ con $u,v\not\in \mathcal U(A)$. Entonces $\mathfrak a\subseteq uA$ y, además, $uA\neq A$ porque $u$ no es unidad. Veamos que $uA \not \subseteq \mathfrak a$, o equivalentemente, $u\not \in \mathfrak a$. Si $u\in \mathfrak a$ existe un $w$ tal que $u = aw =u(vw)$ y por tanto $u(1-vw)=0$ luego $1=vw$, ya que $u\neq 0$ pues si no $\mathfrak a = 0$ que no es maximal. Esto va en contra de la suposición de que $v \not \in \mathcal U(A)$. Así que $\mathfrak a \subsetneq uA\subsetneq A$ y por tanto no es un ideal maximal.

Supongamos ahora que $a$ es irreducible, y existe $\mathfrak b=bA \supset \mathfrak a$. Existe $w \in A$ tal que $a = bw$, y como $a$ es irreducible entonces $b\in \mathcal U(A)$ o $w\in \mathcal U(A)$, en cualquier caso $\mathfrak b = A$, y por tanto $\mathfrak a$ es maximal.

\paragraph{Ejercicio 6}\textit{
\begin{enumerate}[label=(\roman*)]
    \item Sea $A$ un anillo, demostrar que existe una biyección entre las descomposiciones $\Phi:A\to A_1 \times\ldots \times A_n$ via un isomorfismo de anillos y los conjuntos de idempotentes ortogonales de $A$, ie. $\{e_1,\ldots,e_n\}\subset A$ tales que $\sum_{i=1}^ne_i = 1_A$ y $e_ie_j  = \delta_{ij}e_i$.
    \item Demostrar que dada una descomposición, los $A_i$ se identifican con ideales de $A$, no con subanillos. ¿Qué descomposición corresponde al conjunto de idempotentes $\{0_A, 1_A\}$.
\end{enumerate}
}

(i) Veamos este apartado de dos formas: una donde los idempotentes son endomorfismos y otra donde son elementos de $A$.

1. Si tenemos $A = A_1\times \dots \times A_n = \bigoplus_{i=1}^n A_i$, entonces podemos tomar la proyección $A\to A_i$ compuesta con la inclusión $A_i \to A$ que resulta en un endomorfismo de $A$ que denotamos $e_i$. Este endomorfismo es idempotente. Efectivamente, si tomamos $x=(x_1,\dots,x_n)\in A = \bigoplus_{i=1}^n A_i$ entonces $e_i\circ e_i(x) = e_i(0,\dots,0,x_i,0,\dots,0) = (0,\dots,0,x_i,0,\dots,0)$. Son ortogonales porque $e_j(0,\dots,0,x_i,0,\dots,0) = (0,\dots, 0)$. Y también tenemos que suman la identidad porque para cualquier $x\in A$:

\begin{multline}
e_1(x)+\ldots+e_i(x)+e_j(x)+\ldots +e_n(x)=\\=(x_1,0,\dots,0)+\dots+(0,\dots,x_i,0,\dots,0)+(0,\dots,0,x_j,\dots,0)+ (0,\dots,0,x_n) =\\= (x_1,\dots,x_i,x_j,\dots,x_n) = x
\end{multline}

​	Por otra parte, si tenemos un subconjunto $\{e_i\}_{i=1}^r$ tal que  $\sum_{i=1}^r e_i = 1$ y $e_ie_j=\delta_{ij}e_i$ podemos definir una descomposición de 	$A$ tomando $A_i$ las imágenes de los $e_i$.

2. Dado el isomorfismo $\Phi:\bigoplus A_i \to A$, este determina un conjunto de idempotentes según a donde envíe a los elementos siguientes:
\begin{align*}
\Phi: A_1\times\ldots \times A_n &\to A\\
(1,0,\dots,0)&\mapsto e_1\\
(0,1,\dots,0)&\mapsto e_2\\
&\vdots\\
(0,0,\dots,1)&\mapsto e_n
\end{align*}

Efectivamente, por ser homomorfismo ha de cumplirse que

\begin{align}
   1_A&=\Phi(1,1,\dots,1) = \Phi(1,0,\ldots,0)+\dots+\Phi(0,0,\ldots, 1) = e_1+e_2+\ldots e_n\\
   0_A &= \Phi(0,0,\dots,0) = \Phi((0,\dots,\overset{i)}{0},\dots,0)\cdot (0,\dots,\overset{j)}{0},\dots,0)) \quad i\neq j\\
   e_i &= \Phi((0,\dots,\overset{i}{1},\ldots,0)\cdot(0,\dots,\overset{i}{1},\ldots,0)) = e_ie_i
\end{align}

Recíprocamente, dados $\{e_i\}_{i=1}^r$ tomemos los ideales $\mathfrak a_i =e_iA$ de $A$. Estos tienen estructura de anillo conmutativo unitario con las operaciones heredadas y tomando $1_{\mathfrak a_i} = e_i$. En efecto, todo el resto de propiedades se cumple automáticamente y comprobamos que esa es la unidad: para todo $x\in \mathfrak a_i$ existe $a\in A$ tal que $x=e_ia$ y entonces $xe_i = e_ix = e_i e_i a = e_i a = x$.

Ahora consideramos $\phi_i:A\to \mathfrak a_i$ dado por $x\mapsto\phi_i(x) = xe_i$ que es un homomorfismo suprayectivo (esto segundo es obvio porque $\mathfrak a_i = e_iA$):

\begin{align}
    \phi_i(x+y) &= (x+y)e_i = xe_i+ye_i = \phi_i(x)+\phi_i(y)\\
\phi_i(xy) &= xye_i = xye_ie_i = (xe_i)(ye_i) = \phi_i(x)\phi_i(y)
\end{align}

Finalmente podemos coger $\Phi:A\to \bigoplus\mathfrak a_i$ como $\Phi=\bigoplus_i\phi_i$ que es homomorfismo suprayectivo por serlo cada una de las coordendas, y además es inyectivo porque si $x\in A$ es tal que $0 = \Phi(x) = (xe_1,\dots,xe_n)$ entonces $0 = \sum_ixe_i  = x\sum_i e_i = x$. Por lo tanto $\Phi$ es el isomorfismo que buscabamos.

(ii) Claramente $A_i \cong 0\times \ldots\times A_i\times \ldots \times 0$ y este es un ideal de $A_1\times \ldots\times A_n \cong A$ lo que demuestra la identificación. Efectivamente dados $a,b\in A_i$, y $(x_1,\ldots,x_n)\in A_1\times \ldots\times A_n$ tenemos

\begin{equation}
(0,\dots,\overset{i)}{a},\ldots,0)-(0,\dots,\overset{i)}{b},\ldots,0) = (0,\dots,\overset{i)}{a-b},\ldots,0)\in 0\times \ldots\times A_i\times \ldots \times 0
\end{equation}
\begin{equation}
(x_1,\ldots,x_n)\cdot (0,\dots,\overset{i)}{a},\ldots,0) = (0,\dots,\overset{i)}{x_ia},\ldots,0) \in  0\times \ldots\times A_i\times \ldots \times 0
\end{equation}

No es un subanillo porque carece del elemento unidad de $A_1\times \ldots \times A_n$ que es la tupla con todo unos.

Finalmente, si tomamos el conjunto de idempotentes ${0_A,1_A}$ obtenemos la descomposición trivial $A = \{0_A\}\times A$. Si seguimos la forma 2. de proceder, el isomorfismo $\Phi:A_1\times A_2 \to A$ debería asignar $(1,0)\mapsto 0_A$ y $(0,1)\mapsto 1_A$.  Está bien definido porque se cumple que $1_A = 0_A+1_A = \Phi(1,0)+\Phi(0,1) =\Phi(1,1)$ como debe ser.


\paragraph{Ejercicio 7} \textit{Encontrar un sistema de idempotentes ortogonales no trivial y una descomposición asociada para
\begin{enumerate}[label=(\roman*)]
    \item $\Z_{nm}$ con $\gcd(n,m) = 1$.
    \item $\Q[X]/\gen{x^2(x-1)}$.
    \item $K[X]/\gen{fg}$ con $\gcd(f,g) = 1$.
\end{enumerate}}

(i) Sabemos que si $m,n$ son coprimos entonces $\mathbb Z_{mn} \cong \mathbb Z_m \times \mathbb Z_n$. Esta es nuestra descomposición. Para sacar los idempotentes ortogonales nos valemos de la identidad de Bezout: por ser coprimos existen $\mu, \nu $ tales que $\mu m+ \nu n = 1_\Z$. Además tenemos que
\begin{align}
[\mu m] +[\nu n] &= [1_\Z] = 1_{\Z_{mn}}\\
[\mu m] [\nu n] &= [\mu \nu] [nm] = [0]\\
[\mu m] [\mu m] &= [\mu m][1-\nu n] = [\mu m]
\end{align}
Por tanto, $e_1 = [\mu m]$ y $e_2 = [\nu n]$ son los elementos que buscamos. La descomposición viene dada por los ideales $[\mu m]\Z_{mn}$ y $[\nu n]\Z_{mn}$. Veamos que son precisamente $\Z_n$ y $\Z_m$ respectivamente. Los elementos del ideal $[\mu m]\Z_{mn}$ son los restos de la división $\frac{\mu m x}{mn} = \frac{\mu x}{n}$, es decir, son restos que determina una clase en $\Z_n$, por tanto $[\mu m]\Z_{mn} \subset \Z_n$. Pero además, si $[x],[y]\in \Z_{mn}$ son tales que $[\mu mx] = [\mu m y]$ en $\Z_{mn}$, entonces $\mu m(x-y)\in mn\Z$ por lo tanto $x-y \in n\Z$. Es decir, que hay exactamente $n$ clases en nuestro ideal, por tanto $[\mu m]\Z_{mn} = \Z_n$.

(ii) $A=\Q[x] / \langle x^2(x-1)\rangle$. Este ejemplo es el mismo que el anterior pero en un anillo de polinomios. En ambos casos tenemos un dominio euclídeo y por tanto una identidad de Bezout para el máximo común divisor. En concreto, $\gcd(x^2,x-1) = 1$ que sale en la primera división $x^2 = x(x-1)+1$ o equivalentemente $x^2+x(1-x) = 1$, y podemos tomar como conjunto de idempotentes ortogonales $\{x^2, x(1-x)\}$ que cumplirán, análogamente a lo dicho en a), que $A = \Q[x]/\langle x^2\rangle \times \Q[x]/\langle x(1-x)\rangle$.

(iii) Literalmente lo mismo que el (ii) pero ahora genérico. Se cumple exactamente lo mismo.

\paragraph{Ejercicio 8}
(a) En primer lugar, veamos que $f(x,y):=x^2-y^2-1$ es irreducible en $\R[x,y]$. Para ver esto, tomemos $(a_1x+a_2y+a_3),(b_1x+b_2y+b_3)\in\R[x,y]$, cuyo producto es:
\begin{align*}
&(a_1x+a_2y+a_3)(b_1x+b_2y+b_3)=\\
a_1b_1x^2+a_2b_2y^2+a_3&b_3+(a_1b_2+a_2b_1)xy+(a_1b_3+a_3b_1)x+(a_2b_3+a_3b_2)y.
\end{align*}
De esto se desprende que, en caso de que $f$ sea reducible, por ser de grado $2$ ambos factores deben tener grado $1$ y sus coeficientes deben verficar las ecuaciones
\begin{align*}
a_1b_1=&\ 1,\\
a_2b_2=&\ 1,\\
a_3b_3=&-1,\\
a_1b_2+a_2b_1=&0,\\
a_1b_3+a_3b_1=&0\ \text{y}\\
a_2b_3+a_3b_2=&0.
\end{align*}
Distingamos casos.\begin{itemize}
	\item Si $a_1>0$, entonces $b_1>0$. Igualmente, si $a_2>0$, $b_2>0$ y $a_1b_2+a_2b_1>0$. De igual forma, si $a_2<0$, $b_2<0$ y $a_1b_2+a_2b_1<0$. En ambos casos llegamos a un absurdo.
	\item Si $a_1<0$, entonces $b_1<0$ y llegamos de forma análoga al caso anterior a los mismos absurdos.
\end{itemize}
Así, tenemos que $f$ es irreducible en $\R[x,y]$. Como $\R[x,y]$ es DFU (basta verlo como $(\R[x])[y]$), tenemos que $\langle f\rangle$ es un ideal primo de $\R[x,y]$. Si $hg\in\langle f\rangle$, tenemos que $hg=\lambda f$ para cierto $\lambda\in\R[x,y]$. Por ser $f$ irreducible y la factorización de 
$$hg:=h_1\cdots h_rg_1\cdots g_s$$
única, $f=h_i$ para cierta $i\in\{1,\dots,r\}$ o $f=g_j$ para cierta $j\in\{1,\dots, s\}$; en el primer caso $h\in\langle f\rangle$ y en el segundo $h\in\langle f\rangle$. 

Por ser $\langle f\rangle$ primo, $A$ es un $DI$. Supongamos $\af=[p]$ para cierto $[p]\in A$. De ser así, se tendría que $[x-1]=[\lambda_1][p]$ y $[y]=[\lambda_2][p]$ para ciertos $[\lambda_1],[\lambda_2]\in A$ y
\begin{align*}
0=[x^2+y^2-1]=[(x+1)(x-1)+y^2]=([x+1][\lambda_1]+[y][\lambda_2])[p]
\end{align*}

(b) Comprobemos ahora que $\langle[x-(1+iy)]\rangle=\bfr$. En primer lugar, teniendo en cuenta 
\begin{align*}
\left(\left[\frac{-1}{2i}\right][x+(1+iy)]\right)[x-(1+iy)]&=\left[\frac{-1}{2i}\right][x^2-(1+iy)^2]=\\
&=\left[\frac{-1}{2i}\right][x^2-1-2iy+y^2]=\\
&=\left[\frac{-1}{2i}\right][-2iy]=[y]\in\langle [x-(1+iy)]\rangle,
\end{align*}
tenemos que
\begin{align*}
[-x-iy][x-1-iy]=[x-x^2-y^2+iy]=[x-1]\in\langle [x-(1+iy)]\rangle.
\end{align*}
Por otra parte, como 
$$[x-1-iy]=[x-1]-i[y]$$
podemos concluir $\langle[x-(1+iy)]\rangle=\bfr$.

\paragraph{Ejercicio 9} \textit{Sea $A$ un anillo y $\mathfrak a \subset A$ un ideal. Denotamos
\[\mathfrak a [X] = \{ f \in A[X] \vert \; f \text{ tiene sus coeficientes en } \mathfrak a\}\]
Demostrar que $\mathfrak a[X]$ es el extendido de $\mathfrak a$ via la inclusión. Si $\mathfrak p$ es ideal primo de $A$, ¿es $\mathfrak p[X]$ un ideal primo de $A[X]$?}

Estamos considerando la extensión de $\mathfrak a$ por la inclusión $i:A\hookrightarrow A[X]$, entonces
$$
\mathfrak a^e = \langle \mathfrak i(a) \rangle \equiv \langle \mathfrak a\rangle_{A[X]} = \left\{\sum_{i=0}^n a_i g_i\big \vert\; a_i \in \mathfrak a, g_i \in A[X],n\in \N   \right\}
$$
Ahora bien, $\sum_{i=0}^n a_i g_i = \sum_{i=0}^n a_i \sum_{j=0}^m b^i_j X^j = \sum_{i,j}(a_ib^i_j) X^j$ y se cumple  $a_ib^i_j \in  \mathfrak a$ para todo $i,j$ por ser un ideal.

% Lo que sigue está mal:
%
% ---
%
% Sea $\mathfrak p$ un ideal primo de $A$. Sean $f,g \in A[X]$ que identificamos con sucesiones $(a_n)_n,(b_n)_n$ donde a partir de algún término son todos nulos. Supongamos $h = fg \in \mathfrak p[X]$, de coeficientes $(c_n)_n$.
%
% Tenemos $a_0b_0 = c_0 \in \mathfrak p$. Supongamos $a_0 \not\in \mathfrak p$. El siguiente coeficiente del producto es $a_0b_1+a_1b_0 = c_1 \in \mathfrak p$. Por ser un ideal $a_1b_0\in\mathfrak p$, y por tanto $a_0b_1 = c_1-a_1b_0 \in \mathfrak p$. Tenemos entonces $b_1 \in \mathfrak p$. Si hubiésemos comenzado al contrario, tendríamos $a_0,a_1 \in \mathfrak p$. Observamos que para todos estos cálculos no hace falta suponer que ninguno de los términos es distinto de $0$.
%
% Suponemos entonces que para todo $k\leq n$ se cumple que $b_k\in \mathfrak p$ y comprobemos que $b_n \in \mathfrak p$. La primera vez que hace su aparición es en la expresión $c_n = \sum_{i+j = n} a_ib_j$. Por hipótesis de inducción $a_0b_n = c_n - \sum_{\substack{i+j = n\\j\neq n}} a_ib_j \in \mathfrak p$
%
% Tomamos el siguiente coeficiente $a_0b_2+a_1b_1+a_2b_0 = c_2 \in\mathfrak p$. Si $a_0 \not \in \mathfrak p$
%
% \paragraph{Ejercicio 10} \textit{Sea $A$ un anillo, $M$ un $A$-módulo y $\mathfrak a$ un ideal contenido en $\Ann (M)$. Demostrar que $M$ tiene estructura de $A/\mathfrak a$-módulo.}
%
% Solo hay que ver que la multiplicación por escalares de $A/\mathfrak a$ está bien definida. Si $x+\mathfrak a = y+\mathfrak a$ entonces $x-y \in \mathfrak a \subset \Ann(M)$, es decir, $(x-y)m = 0$ para todo $m\in M$, o equivalentemente $xm = ym$ para todo $m\in M$, lo que implica que $(x+\mathfrak a)m = (y+\mathfrak a)m$, y así el producto externo está bien definido.

\paragraph{Ejercicio 11} \textit{Sea $A$ un anillo, $\mathfrak a$ un ideal, y $\mathfrak p_1, \ldots, \mathfrak p_n$ ideales primos. Si $\mathfrak a \subset \bigcup_{i=1}^n \mathfrak p_i$, entonces $\mathfrak a \subset \mathfrak p_i$ para algún $i\in \{1, \ldots, n\}$.}

Probamos el contrarrecíproco por inducción sobre $n$. El caso $n=1$ es obvio. Supongamos que si tenemos $n$ ideales primos y $\mathfrak a \not \subset \mathfrak p_i$ para ningún $i$, entonces $\mathfrak a \not \subset \bigcup_{i=1}^n \mathfrak p_i$, y estudiamos el caso $n+1$.  Vamos a encontrar un elemento de $\mathfrak a$ que no pertenece a ningún $\mathfrak p_i$.

Para cada $j$ consideramos un $z_j \in \mathfrak a \setminus \bigcup_{i\neq j} \mathfrak p_i \neq \varnothing$. La diferencia conjuntista es efectivamente no vacía por hipótesis de inducción, pues hay $n$ ideales primos en esa unión. Además, podemos suponer que $z_j \in \mathfrak p_j$ para cada $j$, pues en caso contrario existe algún $z_j$ que no pertenece a ninguno de los ideales primos y hemos terminado. Afirmamos que el elemento $z= z_1\cdot\ldots\cdot z_n + z_{n+1}\in \mathfrak a$ no pertenece a la unión.

Si perteneciese, a algún $\mathfrak p_j$ para $j\leq n$, entonces $z_{n+1} = z_j -z_1\cdot\ldots\cdot z_n  \in \mathfrak p_j $, en contra de la construcción. Por otro lado, si $z\in \mathfrak p_{n+1}$, entonces $z_1\cdot\ldots\cdot z_n = z-z_{n+1} \in \mathfrak p_{m+1}$ y por ser este un ideal primo alguno de los $z_i$, con $1 \leq i \leq n$, pertenece a $\mathfrak p_{n+1}$, de nuevo en contra de la construcción de $z$.

\paragraph{Ejercicio 13} Sea $A$ un anillo e $I\subset A[X_1,\ldots, X_n]$ un ideal. Demostrar que $A[X_1,\ldots, X_n]/I \cong A$ y que si $A$ es un cuerpo, $I$ es maximal.

La última afirmación es evidente, porque un ideal es maximal si y solo si el cociente es un cuerpo.  Para ver el isomorfismo solo hace falta coger el homomorfismo suprayectivo $\operatorname{eval}_{a_1,\ldots, a_n}:A[X_1,\ldots,X_n]\to A$ cuyo núcleo son los polinomios de la forma $\sum_i (x_i-a_i)f$, pues todos sus términos deben anularse, y entonces $\ker \operatorname{eval}_{a_1,\ldots, a_n} = I$ y hemos terminado.

\paragraph{Ejercicio 15}

Se trata de repetir las demostraciones sobre extensiones finitas de cuerpos y la algebricidad de los generadores.

$\Rightarrow$) Si $A$ es un $K$-espacio vectorial de dimensión finita $m$, entonces para cada $i$ las potencias $1, x_i, \ldots, x_i^m$ son $m+1$ vectores del espacio y por tanto son linealmente dependientes. Esto implica que existen $\lambda^i_0, \dots, \lambda^i_m \in K$ tales que $\lambda^i_0 + \lambda^i_1x_i + \ldots+\lambda^i_mx_i^m = 0$, es decir, que el polinomio no nulo $f_i(T) = \lambda^i_0 + \lambda^i_1 T + \ldots+\lambda^i_m T^m \in K[T]$ tiene a $x_i$ por raíz.

$\Leftarrow$) Lo probamos por inducción. Escribimos solo el caso base $A=K[x_1]$. Consideramos el homomorfismo evaluación $\operatorname{eval}_{x_1}:K[T]\to A$. El núcleo $\ker \operatorname{eval}_{x_1}$ es un ideal primo de $K[T]$. Efectivamente, si $f, g \in K[T]$ son tales que $0 = fg (x_1) = f(x_1)g(x_1)$ entonces por ser $A$ un DI, $f(x_1)=0$ ó $g(x_1) = 0$, como queríamos probar. Por ser $K$ un cuerpo, $K[T]$ es un DIP (es dominio euclídeo) y así $\ker \operatorname{eval}_{x_1}$ es un ideal maximal, está generado por un elemento irreducible $f$, y entonces por la caracterización de maximales $K[T]/\gen{f} \cong \im \operatorname{eval}_{x_1}$ es un cuerpo. Dado que la imagen es un cuerpo que contiene a $K$ y a $x_1$ y está contenida en $A$, debe coincidir con A.

Tomamos $f$ el único polinomio mónico irreducible que genera el núcleo. Resulta que el grado  $n$ de $f$ es la dimensión de $K[x_1]$. Efectivamente, $1+\gen{f}, \dots, T^{n-1}+\gen{f}$ es una base de $K[T]/\gen{f}$ (demostración en el libro de Gamboa). Además el isomorfismo $g+\gen{f} \mapsto g(x_1)$ entre $K[T]/\gen{f}$ e $\im \operatorname{eval}_{x_1}$ es un isomorfismo de $K$-espacios vectoriales porque deja fijos todos los elementos de $K$. Entonces $1, x_1, \ldots, x_1^{n-1}$ es una base de $A=K[x_1]$.

\paragraph{Ejercicio 17} \textit{Sea $A$ un anillo y $f, g\in A[T]$ dos polinomios primitivos. Probar que $fg$ es un polinomio primitivo.}

Supongamos que $fg$ no es primitivo. Entonces el ideal $\mathfrak a$ que generan sus coeficientes no es el total. Sea $\mathfrak m$ un ideal maximal que contiene a $\mathfrak a$.

Consideramos $A/\mathfrak m [T] \cong (A/\mathfrak m)[T]$. Esto es cierto, podemos definir el homomorfismo suprayectivo $A[T] \to (A/\mathfrak m)[T]$ dado por $f= \sum a_iT^i \mapsto \sum (a_i+\mathfrak m) T^i$, cuyo núcleo es $\mathfrak m[T]$. Ese cociente es un cuerpo, en particular un dominio de integridad. La clase de $fg$ se anula en el cociente, pero no las clases de $f, g$ porque sus coeficientes generan todo $A$, y si se anulasen significaría que $\mathfrak m = A$. Y esto es absurdo porque $[fg] = [f][g]$.

\paragraph{Ejercicio 18} \textit{Sea $A$ un anillo y $M$ un $A$-módulo. Definimos en $A\times M$ la multiplicación $(a,m)(b,n) = (ab,an+bm)$ con la suma natural y el producto de $A$-módulo. Probar que $A\times M$ es una $A$-álgebra con la suma natural y ese producto. ¿Es el homomorfismo $a\mapsto (a,0_M)$ inyectivo?}

Para ver que es $A$-álgebra solo hay que demostrar que $A\times M$ es un anillo (conmutativo unitario). Como $(A,+)$ y $(M,+)$ son grupos abelianos, $(A\times M, +)$. donde la suma es por coordenadas, también es un grupo abeliano.

El producto es conmutativo $(b,n)(a,m) = (ba,bm+an) = (ab,an+bm) = (a,m)(b,n)$ y distributivo:
\begin{multline}
  (a,m)[(b,m)+(c,k)]=(a,m)(b+c,m+k) = \left(a(b+c),a(n+k)+(b+c)m\right) = (ab+ac,an+ak+bm+cm) =\\ (ab,an+bm)+(ac,ak+cm) = (a,m)(b,n)+(a,m)(c,k)
\end{multline}
y tiene unidad $(a,m)(1_A,0) = (a1_A,a0+1_A m) = (a,m)$.


Obviamente la inclusión de un factor en un producto cartesiano es siempre inyectiva.

\paragraph{Ejercicio 19}

\subparagraph{Ejercicio 17 del Atiyah}

Comprobamos las dos condiciones para ser base. En primer lugar $\bigcup_{f\in A}X_f = \bigcup_{f\in A}\Spec A \setminus V(f) = \Spec A \setminus \bigcap_{f\in A}V(f) = \Spec A$. Esto último es porque $V(f)\cap V(g) = V(\{f,g\})$ para cualesquiera $f,g \in A$, luego $\bigcap_{f\in A}V(f) = V(A) = V(\gen{1}) = \varnothing$. En segundo lugar, sean $f, g \in A$ y $\mathfrak p \in X_f\cap X_g = \Spec A \setminus (V(f) \cup V(g))$. Entonces $f, g\not\in \mathfrak p$, y por ser primo $fg \not \in \mathfrak p$, luego $\mathfrak p \in X_{fg}$. Y si $\mathfrak q \in X_fg$, entonces $fg \not \in \mathfrak q$, y por ser ideal esto implica que $f \not \in \mathfrak q$ y $g \not \in \mathfrak q$, por tanto $X_fg \subset X_f \cap X_g$, lo que termina la demostración de que ese conjunto es base de la topología. Además tenemos los dos contenidos lo que demuestra (i).

(i) $X_f \cap X_g = X_{fg}$.

(ii) $\varnothing = \Spec A \setminus V(f) \iff V(f) = \Spec A \iff f \in \bigcap_{\mathfrak p \in \Spec A} \mathfrak p = \mathfrak N_A$.

(iii) Sabemos que si $f\not\in \mathcal U(A)$, entonces existe un ideal maximal que lo contiene, en particular existe un ideal primo que lo contiene. Luego si ningún ideal primo lo contiene, no existe maximal que lo contenga, entonces no es unidad:  $\varnothing = V(f) \Rightarrow pf \not \in \mathfrak p \forall \mathfrak p \in \Spec A \Rightarrow f \not\in \mathcal U(A)$. Por otra parte, si $f$ es unidad, no puede estar contenido en ningún ideal que no sea el total, y por tanto no hay primo (un ideal propio) que lo contenga.

(iv) $X_f = X_g \iff V(f) = V(g)$, y $\gen{f}$ es el menor radical que contiene a $f$, luego $\forall \mathfrak p \in V(f)$ se tiene $\gen{f} \subset \mathfrak p$ y que $\sqrt{\gen{f}} = \bigcap_{\mathfrak p \in V(f)} \mathfrak p = \bigcap_{\mathfrak p \in V(g)} \mathfrak{p} = \sqrt{\gen{g}}$.
Recíprocamente, si $\bigcap_{\mathfrak p \in V(f)} \mathfrak p = \bigcap_{\mathfrak q \in V(g)} \mathfrak{q}$, dado $\mathfrak p \in V(f)$, $ \bigcap_{\mathfrak{q} \in V(g)} \mathfrak{q} \subset \mathfrak{p}$ y por ende $g \in \mathfrak p$ luego $\mathfrak p \in V(g)$; el otro contenido es análogo. Luego $V(f) = V(g)$ y por tanto $X_f = X_g$.

(v) Basta comprobarlo para un recubrimiento por abiertos de la base. Sea $\{X_{f_i}\}_{i\in I}$ recubrimiento de $\Spec A$, y comprobemos que $\gen{\{f_i\}_{i\in I}} = \gen{1}$. Efectivamente, como $\Spec A = \bigcup{i\in I}X_{f_i} $, entonces

\begin{equation}
  \varnothing = \bigcap_{i\in I} V(f_i) = V(\{f_i\}_{i\in I}) = V(\gen{\{f_i\}_{i\in I}})
\end{equation}

lo que quiere decir que no hay ningún primo que contenga a $\gen{\{f_i\}_{i\in I}}$, en particular no hay ningún maximal que lo contenga, es decir, que $\gen{\{f_i\}_{i\in I}} = \gen{1}$. Entonces existe $J\subset I$ finito y existen $\set{\lambda_j}_{j\in J}$ tales que $1 = \sum_{j\in J} \lambda_j f_j$.
Por tanto $\gen{ \set{f_j}_{j\in J}} = \gen{ 1}$ y así  $V(\gen{\{f_j\}_{j\in J}} )= \varnothing$ lo que implica $\bigcup_{j\in J} X_{f_j} = \Spec A$. Con lo que $\set{X_{f_j}}_{j\in J}$ es subrecubrimiento finito de $\{X_{f_i}\}_{i\in I}$.

(vi) Consideramos $(X_{g_i})_{i\in I}$ recubrimiento de $X_f$. Podemos suponer spg. que $X_f = \bigcup_{i\in I} X_{f_i}$ por ser abierto. Entonces tenemos $V(f) = V(\gen{f_i}_{i\in I})$ y por tanto $f\in \sqrt{\gen{f_i}_{i\in I}}$ de forma que existe un $n>0$ tal que $f^n \in \gen{f_i}_{i\in I}$.
Por tanto, existe $J\subset I$ finito y $\set{a_j}_{j\in J}$ tales que $f^n = \sum_{j\in J}a_j f_j$.

Esto implica que para todo $\mathfrak p \in  V(\gen{f_j}_{j\in J})$ se cumple $\gen{f} \subset \mathfrak p$, y a su vez $f \in \mathfrak p$, de manera que $ V(\gen{f_j}_{j\in J}) \subset V(f)$. Los complementarios cumplen la inclusión contraria

\[ X_f = \Spec A \setminus V(f) \subset \Spec A \setminus V(\gen{f_j}_{j\in J}) = \bigcup_{j\in J} X_{f_j} \]

y por tanto $\set{X_{f_j}}_{j\in J}$ es un subrecubrimiento finito.

(vii) $\Rightarrow)$ Supongamos que $A$ es abierto y compacto. Por ser abierto es unión de abiertos de la base, $A=\bigcup_{i\in I}X_{f_i}$, estos forman un recubrimiento y por ser compacto podemos quedarnos con un subrecubrimiento finito: $A=\bigcup_{i=1}^n X_{f_i}$.

$\Leftarrow)$ Si $A=\bigcup_{i=1}^n X_{f_i}$, entonces es abierto por ser unión de abiertos. Sea $(X_{g_j})_{j\in J}$ un recubrimiento de $A$, en particular recubren cada $X_{f_i}$. Para cada $i=1,\dots, n$ por ser compacto existe $F_i \subset J$ finito tal que $X_{f_i}\subset \bigcup_{j \in F_i} X_{g_j}$. Por tanto $A \subset \bigcup{i=1}^n \bigcup_{j \in F_i} X_{g_j}$.

\end{document}
