\documentclass[../main.tex]{subfiles}
\begin{document}
\section{Recodatorio teoría de cuerpos}
Sea $K \subset L$ una extensión de cuerpos. Un elemento $u \in L$ es \emph{transcendente} sobre $K$ si ningún polinomio no nulo de $K[T]$ se anula sobre este, es decir, $p(u)=0 \Rightarrow p=0 \in K[T]$.

Dados $\left\{u_{1}, \ldots, u_{s}\right\} \subset L$ se dice que son \emph{algebraicamente independientes} sobre $K$ si no hay ningún polinomio en $p\in K[T_1,\dots, T_s]$ no nulo que tal que $p\left(u_{1}, \ldots, u_{s}\right)=0$.
Se desprende de la definición que si $\{u_{1}, \ldots, u_{s}\}$ son algebraicamente independientes entonces el menor subcuerpo de $L$ que contiene a $K$ y a los $u_i$, que notaremos $K(u_{1}, \ldots, u_{s})$, es isomorfo a un anillo de polinomios:
$K(u_{1}, \ldots, u_{s}) \cong K[T_{1}, \ldots, T_{s}]$.

Sea $K \subset L=K\left(\theta_{1}, \ldots, \theta_{n}\right)$ una extensión de cuerpos finitamente generada, entonces existe $\{\theta_{i_{1}}, \ldots, \theta_{i_{r}}\}\subset \{\theta_{1}, \ldots, \theta_{n}\}$ algebraicamente
independientes sobre $K$ tales que $K\left(\theta_{i_{1}}, \ldots, \theta_{i_{r}}\right) = L$. El conjunto $\left\{\theta_{i_{1}}, \ldots, \theta_{i_{r}}\right\}$ se denomina \emph{base de trascendencia}. Su cardinal es común a todas las bases de trascendencia y se denomina \emph{grado de trascendencia} de $L$ sobre $K$.

Si $A \subset B$ son D.I se puede hablar de grado de trascendencia de $B$ sobre $A$ como el correspondiente a la extensión entre sus respectivos cuerpos de fracciones.

\begin{remark}[\textbf{(Notación)}]
Durante el capítulo, las letras mayúsculas denotarán elementos algebraicamente independientes.
\end{remark}

\section{Lema de normalización de Noether}
\begin{lemma}
Sea $K$ un cuerpo infinito y $F\in K[X_1, \dots, X_n]$ un polinomio homogéneo no nulo. Entonces existe $(a_1, \dots, a_{n-1}) \in K^{n-1}$ tales que $F(a_1, \dots, a_{n-1},1)\neq 0$.
\end{lemma}
\begin{proof}
Por inducción sobre $n$. El caso para $n=1$ es trivial porque $F(X_1) = aX_1$ con $a\neq 0$. Suponemos el resultado cierto para $n>1$ y escribimos $F = \sum_{k=0}^d F_k X_1^k$ donde $F_k \in K[X_2,\dots, X_n]$ son polinomios homogéneos de grado $d-k$.  Si $F\neq 0$, al menos algún $F_k$ es no nulo.
Por inducción podemos escoger $(a_2, \dots, a_{n-1}) \in K^{n-2}$ tal que $F_k(a_2, \dots, a_{n-1},1)\neq 0$. Entonces $F(.,a_2, \dots, a_{n-1},1) \in K[X_1]$ es un polinomio no nulo y tiene por tanto un número finito de raíces. Como $K$ es infinito, existe algún $a_1 \in K$ tal que $F(a_1,a_2, \dots, a_{n-1},1) \neq 0$.
\end{proof}
\begin{remark}\label{obs_import}
Sea $K\left[X_{1}, \ldots, X_{n}\right]$ el anillo de polinomios sobre un cuerpo $K$ infinito,  y sea $F \in K\left[X_{1}, \ldots, X_{n}\right] \setminus K$ un polinomio no constante. Podemos escribir $F$ en componentes homogéneas: $F=F_{r}+\ldots+F_{s}$ donde el subíndice indica el grado de la componente, con $r \leq s$ y con $F_{s} \neq 0 .$

Por el lema, existe $\left(\alpha_{1}, \ldots, \alpha_{n-1}\right) \in K^{n-1}$ tal que $F_{s}\left(\alpha_{1}, \ldots, \alpha_{n-1}, 1\right) \neq 0$. Realizamos el siguiente cambio de variable
$$
\begin{cases}
  Y_{i}:=X_{i}-\alpha_{i} X_{n} & i=1 \dots n-1\\
  Y_n := X_n
\end{cases}
$$
Entonces, podemos sustituir el cambio en $F$ y separar en componentes homogéneas:
\begin{multline}
  F\left(Y_{1}+\alpha_{1} X_{n}, \ldots, Y_{n-1}+\alpha_{n-1} X_{n}, X_{n}\right)=\\
  =\left (\sum_{i_{1}+\cdots+i_{n}=s} a_{i_{1} i_{2} \ldots i_{n}}\left(Y_{1}+\alpha_{1} X_{n}\right)^{i_{1}} \left(Y_{1}+\alpha_{1} X_{n}\right)^{i_{2}} \ldots \left(Y_{n-1}+\alpha_{n-1} X_{n}\right)^{i_{n-1}}  X_{n}^{i_{n}} \right )+ G = \\
  =F_{s}\left(\alpha_{1}, \ldots, \alpha_{n-1}, 1\right) X_{n}^{s}+ G = cX_n^s+G
\end{multline}
donde $G$ agrupa los términos de menor grado en $X_n$. De otra forma, $c X_{n}^{s}+ G-F=0$ y acabamos de comprobar que $A:=K\left[Y_{1}, \ldots, Y_{n-1}, F\right] \subset K\left[X_{1}, \ldots, X_{n}\right]$ es una extensión entera, ya que $X_{n}$ es entero sobre $A$ y $\left\{Y_{1}, \ldots, Y_{n-1}, F\right\}$ son algebraicamente independientes sobre $K$.
\end{remark}

\begin{theorem}[\textbf{(de normalización)}]
Sea $K$ cuerpo infinito, $A=K\left[x_{1}, \ldots, x_{n}\right]$ una $K$-álgebra finitamente generada\footnote{Es útil pensarlo como un cociente del anillo de polinomios por un ideal.} e $I$ un ideal propio de $A$. Entonces existen $d \leq n, \delta \leq d, Y_{1}, \ldots, Y_{d} \in A$ algebraicamente independientes
sobre $K$ tal que
\begin{enumerate}
  \item la extensión $A \supset A^{\prime}:=K\left[Y_{1}, \ldots, Y_{d}\right]$ es entera; ie. A es un $K\left[Y_{1}, \ldots, Y_{d}\right]$ -módulo finito.
  \item $I \cap K[Y_{1}, \ldots, Y_{d}]=\left\langle Y_{\delta+1}, \ldots, Y_{d}\right\rangle$,
  \item los $Y_{1}, \ldots, Y_{\delta}$ son combinaciones lineales de $\operatorname{los} x_{1}, \ldots, x_{n}$ con coeficientes en $K$.
\end{enumerate}
\end{theorem}

\begin{remark}
Si $\delta=d$, significa $I=\langle 0\rangle$. Además, $d=0$ cuando $A$ es una extensión algebraica de $K$.
\end{remark}

\begin{remark}
Cuando $I=\langle 0\rangle$,se predican (i) y (ii) se tiene el \emph{Lema de normalización de Noether} usual.
\end{remark}

\begin{proof}
Distinguiremos varios casos:

\paragraph{Caso 1:} Supongamos que los generadores de $A$ son algebraicamente independientes, ie. $A\cong K[X_1, \dots, X_n]$, y que el ideal es principal $I = \langle f \rangle$. Podemos hacer el cambio de variable de la observación \ref{obs_import} y así sabemos que $A':= K[Y_1, \dots, Y_{n-1}, Y_n] \subset K[X_1, \dots, X_n]$ es una extensión entera, donde hemos llamado $Y_n := f$. Como el grado de trascendencia de $K[X_1, \dots, X_n]$ sobre $K$ es $n$ por ser algebraicamente independientes, tenemos que $Y_1, \dots, Y_n$ son algebraicamente independientes.

Entonces $A'$ es un anillo de polinomios con coeficientes en un cuerpo, luego es un $DFU$ y por tanto es integramente cerrado sobre su cuerpo de fracciones.

Comprobamos que $A' \cap I = \langle Y_n \rangle_{A'}$. Recordamos que $Y_n = f$, luego el contenido $\supset$ es automático. Sea ahora $h = g f \in I \cap A'$ (es de esa forma por pertenecer a $I$). Entonces $g$ pertence a $K_{A'}$ el cuerpo de fracciones de $A'$ y es entero sobre $A'$, por lo tanto $g \in A'$ así efectivamente $h$ es un múltiplo de $Y_n$ en $A'$.

\paragraph{Caso 2:} Supongamos de nuevo que $A= K[X_1,\dots, X_n]$ es un anillo de polinomios y ahora tomemos un ideal $I$ arbitrario. Por inducción sobre $n$. Si $n=1$, $K[X_1]$ es DIP y estamos en caso 1. Supongamos el resultado cierto para todo $k< n$ y lo probamos para $n$.

Si $I=\langle 0\rangle$ es trivial. En otro caso, sea $F \in I \backslash\{0\}$, y realizando la construcción anterior obtenemos la extensión entera $B:=K\left[Y_{1}, \ldots, Y_{n-1}, Y_{n}=F\right] \subset A$ que cumple que $I \cap B \supset\langle Y_{n}\rangle_{B}$.

Sea $A_{1}:=K[Y_{1}, \ldots, Y_{n-1}]$ y el ideal
$I_{1}:=I \cap A_{1}$. Por la hipótesis de inducción existen $d'$ y $\delta'$ con $\delta' \leq d'$ tales que
\begin{itemize}[]
  \item[(a)] $K\left[T_{1}, \ldots, T_{d^{\prime}}\right] \subset A_{1}$ es una extensión entera,
  \item[(b)] $I_{1} \cap K\left[T_{1}, \ldots, T_{d^{\prime}}\right]=\left\langle T_{\delta^{\prime}+1}, \ldots, T_{d}^{\prime}\right\rangle, \mathrm{y}$,
  \item[(c)] $T_{1}, \ldots, T_{\delta^{\prime}}$ son
  combinaciónes lineales de $Y_{1}, \ldots, Y_{n-1}$
\end{itemize}

Por (a) tenemos $d^{\prime}=n-1$ y
$$A^{\prime}:=K\left[T_{1}, \ldots, T_{n-1}, Y_{n}\right] \subset K\left[Y_{1}, \ldots, Y_{n-1}, Y_{n}\right] \subset A$$
donde $A$ es $A^{\prime}$ módulo finito, por transitividad.

Tomemos $d:=n$, $\delta:=\delta^{\prime}$, y $T_{n}:=Y_{n}$. Es claro que $T_{\delta^{\prime}+1}, \ldots, T_{n-1}, Y_{n} \in I \cap A^{\prime} $. Asimismo si
$g \in I \cap A^{\prime}$, entonces $$g=g_{1}\left(T_{1}, \ldots, T_{n-1}\right)+Y_{n} g_{2}\left(T_{1}, \ldots, T_{n-1}, Y_{n}\right)$$
Como $Y_{n} \cdot g_{2} \in I \cap A^{\prime}$, se tiene que $$g_{1} \in I \cap K\left[T_{1}, \ldots, T_{n-1}\right]=I_{1} \cap K\left[T_{1}, \ldots, T_{n-1}\right]=\left\langle T_{\delta^{\prime}+1}, \ldots, T_{d^{\prime}}\right\rangle$$
Así $\left\langle T_{\delta^{\prime}+1}, \ldots, T_{n-1}, Y_{n}\right\rangle_{A^{\prime}}=I \cap A^{\prime}$.

Finalmente, $T_1, \dots, T_{\delta'}$ son combinaciones lineales de los $Y_1, \dots, Y_{n-1}$ que a su vez era combinaciones lineales de $X_1, \dots, X_n$, y tenemos el resultado.

\paragraph{Caso general:} Sea $A=K\left[x_{1} \ldots, x_{n}\right]$ e $/$ un ideal. Por caracterización general de álgebras finitamente generadas, existen ideales $J, I_0 \subset K[X_1, \dots, X_n]$ con $J \subset I_0$ tales que  $A \cong K\left[X_{1}, \ldots, X_{n}\right] / J$ e  $I=I_{0} / J$.

Aplicamos el caso 2 a $K\left[X_{1}, \ldots, X_{n}\right]$ y $J$. Se obtienen así $d'=n$, $\delta' \leq n$ y $Y_1, \dots Y_n \in A$ tales que
\begin{itemize}
  \item $K\left[Y_{1}, \ldots, Y_{n}\right] \subset K\left[X_{1}, \ldots, X_{n}\right]$ extensión entera,
  \item $J \cap K\left[Y_{1}, \ldots, Y_{n}\right]=\left\langle Y_{\delta^{\prime}+1}, \ldots, Y_{n}\right\rangle$
  \item $Y_{1}, \ldots, Y_{\delta^{\prime}}$ son combinaciones lineales de los $X_{i}$.
\end{itemize}

Así $K\left[Y_{1}, \ldots, Y_{\delta^{\prime}}\right] \cap J=\langle 0\rangle$ y $K\left[Y_{1}, \ldots, Y_{\delta^{\prime}}\right] \hookrightarrow A=K\left[X_{1}, \ldots, X_{n}\right] / J$ es una extensión entera.

Sea $d:=\delta'$. Aplicando el caso 2 a $A_{1}:=K\left[Y_{1}, \ldots, Y_{d}\right]$ y a $I_{1}:=I_{0} \cap K[Y_{1}, \ldots, Y_{d}]$ existen $\delta'', d''$ con $\delta'' \leq d''$, y existen $T_1, \dots, T_{d''} \in A_{1}$ de manera que
\begin{itemize}
  \item $K\left[T_{1}, \ldots, T_{d^{\prime \prime}}\right] \subset A_{1}$ es extensión entera,
  \item $I_{1} \cap K\left[T_{1}, \ldots, T_{d^{\prime \prime}}\right]=\left\langle T_{\delta^{\prime \prime}+1}, \ldots, T_{d^{\prime \prime}}\right\rangle$\footnote{Observamos que $d^{\prime \prime}=d$.},
  \item $T_{1}, \ldots, T_{\delta^{\prime \prime}}$ son combinaciones lineales de $Y_{1}, \ldots, Y_{d}$
\end{itemize}

Con todo esto, finalmente tenemos que
\begin{enumerate}
  \item  $K\left[T_{1}, \ldots, T_{d}\right] \subset A_{1} \hookrightarrow A$ son extensiones enteras,
  \item  $I \cap K\left[T_{1}, \ldots, T_{d}\right]=\left(I_{0} \cap K\left[Y_{1}, \ldots, Y_{d}\right]\right) \cap K\left[T_{1}, \ldots, T_{d}\right]=\left\langle T_{\delta^{\prime \prime}+1}, \ldots, T_{d}\right\rangle$,
  \item las $T_1, \dots, T_{\delta''}$ son combinaciones lineales de las $X_1, \dots, X_n$.
\end{enumerate}
como queríamos demostrar.
\end{proof}


\section{Nullstellensatz}
\begin{theorem}[forma débil]
	Sea $K$ un cuerpo y $A:=K[x_1,\dots,x_n]$ una $K$-álgebra finitamente generada que es un cuerpo, entonces $A$ es una extensión algebraica de $K.$
\end{theorem}

\begin{proof}Aplicamos a $A$ el lema de normalización con ideal $\langle 0\rangle$, y existen $Y_1,\dots,Y_d\in A$, algebraicamente independientes sobre $K$ (y combinaciones lineales de las variables $x_i$ si $K$ es infinito), tales que $K[Y_1,\dots,Y_d]\subset A$ es entera. Como $A$ es cuerpo, $K[Y_1,\dots,Y_d]]$ es cuerpo. Pero si $d\neq 0$, $K[Y_1,\dots,Y_d]$ es un anillo de polinomios y $\frac{1}{Y_1}\notin K[Y_1,\dots,Y_d]$ es un anillo de polinomios y $\frac{1}{Y_1}\notin K[Y_1,\dots,Y_d]$, que es absurdo. Así, necesariamente $d=0$ y $A$ es extensión algebraica de $K.$
\end{proof}

\begin{theorem}[forma débil]
	Sea $K$ cuerpo algebraicamente cerrado $K[X_1,\dots,X_n]$ el anillo de polinomios y $\mathfrak{m}$ un ideal maximal, entonces existe $(a_1,\dots,a_n)\in K^n$ tal que $\mathfrak{m}=\langle X_1+a_1,\dots,X_n-a_n\rangle.$
\end{theorem}

\begin{proof}
	Por lo anterior, $K\hookrightarrow K[X_1,\dots,X_n]/\mathfrak{m}$, $c\mapsto c+\mathfrak{m}$ es una extensión algebraica, pero como $K$ es algebraicamente cerrado la aplicación es suprayectiva. Así, para $X_i+\mathfrak{m}$ existe $a_i\in K$ tal que $a_i+\mathfrak{m}=X_i+\mathfrak{m}$, es decir, $X_i-a_i\in\mathfrak{m}.$ Como $\mathfrak{m}$ es un ideal maximal de $K[X_1,\dots, X_n]$ y $\mathfrak{m}\supset\langle X_1-a_1,\dots, X_n-a_n\rangle$, que también es maximal, ambos coinciden.
\end{proof}

\begin{theorem}[forma usual]
	Sea $K$ un cuerpo algebraicamente cerrado, $K[x_1,\dots,X_n]$ un anillo de polinomios e $I$ un ideal propio. Entonces $\mathcal{Z}(I)\neq\varnothing$ y
	$$\mathcal{I}(\mathcal{Z}(I))=\sqrt{I}.$$
\end{theorem}

\begin{proof}
	Sea $\mathfrak{m}\supset I$ un ideal maximal de $K[X_1,\dots,X_n]$ que contiene a $I.$ En primer lugar, $\mathcal{Z}(I)\supset\mathcal{Z}(\mathfrak{m})$ por el teorema débil.
	
	Veamos el segundo contenido. Por el teorema de la base, $I=\langle f_1,\dots,f_r\rangle$, para ciertos $f_i\in K[X_1,\dots,X_n].$ Sea $g(X_1,\dots,X_n)\in\mathcal{I}(\mathcal{Z}(I))$ y sea $T$ una nueva variable. En $K[X_1,\dots,X_n,T]$ consideremos el ideal $I_1:=\langle f_1,\dots,f_r,Tg-1\rangle.$ Como $g\in\mathcal{I}(\zeros I)$, $\zeros {I_1}=\varnothing$ en $K^{n+1}.$ Por la primera parte $1\in I_1$; es decir, existen $H_i\in K[X_1,\dots,X_n,T]$ para $i\in\{1,\dots,r+1\}$ tales que
	$$1=\sum_{i=1}^rH_i f_i+H_{r+1}(Tg-1)$$
	Ésta es una identidad polinomial y, evaluando en $T=\frac{1}{g}$ en $K(X_1,\dots,X_n):=K_{K[X_1,\dots,X_n]}$ y operando, tenemos $g^N=\sum_{i=1}^rH_i^*f_i$ para cierto $N\in \N$ y $H_i^*\in K[X_1,\dots,X_n]$; i.e., $g\in\sqrt I.$
\end{proof}

\section{Dimensión de Krull de una $K$-ágebra finitamente generada.}
\begin{theorem}
	Sea $A$ una $K$-álgebra finitamente generada, $A:=K[x_1,\dots,x_n]$, y $K[Y_1,\dots,Y_d]\subset A$ una normalización de Noether. Bajo estas circunstancias $\dimk A=d.$ En particular, si $A$ es dominio de integridad, $\dimk A=\operatorname{grtr}_K A.$
\end{theorem}

\begin{proof}
	Sabemos por \textit{going-up} que $\dimk A=\dimk {K[Y_1,\dots,Y_n]}.$ Tenemos que ver que dicha dimensión es $d.$ Si $d=0$, claramente la dimensión de Krull de un cuerpo es $0.$ Así, podemos suponer que $d\ge 1.$ Además, en tal caso, dicha dimensión es $\ge d$ pues $\langle 0\rangle\subsetneq\langle Y_1\rangle\subsetneq\cdots\subsetneq\langle Y_1,\dots, Y_n\rangle.$
	
	Sea $\p_0\subsetneq\p_1\subsetneq\cdots\subsetneq\p_m$ una cadena estricta de ideales primos en $K[Y_1,\dots,Y_d$ y vemos que $m\le d.$ Si $m=0$, el resultado es claro. Si $m\ge 1$, $\p_1\neq 0.$ Así, por el teorema de normalización existen $d'(=d), \delta\le d$, y $T_i\in K[Y_1,\dots,Y_d]$ tales que $A_1':=K[T_1,\dots,T_d]\subset A_1:=K[Y_1,\dots Y_d]$ es entera y $\p_1\cap K[T_1,\dots, T_d]=\langle T_{\delta+1},\dots,T_d\rangle$, $\delta\le d-1$, pues $\p_1\neq 0.$ 
	
	Considero los ideales $\p_i':=\p_i\cap K[T_1,\dots,T_d]$ que, por ser $A_1'\subset A_1$, entera mantienen la relación de contenido estricto. Tomando cociente por $\p_1$,
	$$\langle 0\rangle\subsetneq \faktor{\p_2'}{\p_1'}\subsetneq\cdots\subsetneq\faktor{\p_m'}{\p_1'}$$
	es una cadena estricta de ideales en $K[T_1,ºdots, T_\delta].$ Por hipótesis de inducción: $m-1\le\delta\le d-1.$ Así, $m\le d.$
\end{proof}

\begin{theorem}
	Sea $A$ una $K$-álgebra finitamente generada, $A=K[x_1,\dots, x_n]$. Si $A$ es dominio de integridad, todas las cadenas maximales de ideales primos de $A$ tienen la misma longitud, $d+1$, siendo $d=\dimk A.$
\end{theorem}

\begin{proof}
	Probemos primero que una cadena estricta y maximal de ideales primos de $A$
	\begin{equation}\label{eq:cadenaprimos1}
	\p_m\supsetneq\cdots\supsetneq\p_0
	\end{equation}
	se contrae en una normalización $A':=K[Y_1,\dots,Y_d]$ a una cadena estricta maximal de ideales primos de $A'$
	\begin{equation}\label{eq:cadenaprimos2}
	\p_m'\supsetneq\cdots\supsetneq\p_0'.
	\end{equation}
	La cadena es estricta por ser $A'\subset A$ extensión entera. Como $A$ es dominio de integridad, $\p_0=\langle 0\rangle$ y $\p_m$ es maximal. Si (\ref{eq:cadenaprimos2}) no es maximal, existe $\p'\subset A'$ ideal primo tal que $\p_{i+1}'\supset\p'\supset\p_i'.$ Si $i=0$, aplicamos \textit{going-down} a la extensión $A'\subset A$ y a los ideales $\p_1\cap A'=\p_1'\supsetneq \p'$ y existe $\p$ ideal primo en $A$ tal que $A'\cap\p'=\p$ y $\p_1\supset\p\neq 0$; pero esto contradice la maximalidad de (\ref{eq:cadenaprimos1}).
	
	Si $i\ge 1$, sea $A'':0K[T_1,\dots, T_d]\subset A'$ entera y $\p_i'\cap A''\langle T_{\delta+1},\dots, T_d\rangle$, con $\delta\le d.$ Llamo $\p_j'':=\p_j'\cap A''$ y $\p'':=\p'\cap A''$ y tenemos extensiones $B':=K[T_1,\dots,T_\delta]\cong A''/\p_i''\hookrightarrow A'/\p_i'\hookrightarrow A/\p_i=:B$ extensiones enteras. De hecho, $B'$ es una normalización de $B.$ Entre $B$ y $B'$ tenemos $\p_{i+1}''/\p_i''=\p_{i+1}'\cap K[T_1,\dots,T\delta]=\p_{i+1}\cap K[T_1,\dots, T_\delta]\supsetneq \p''/\p_i''=\p'\cap K[T_1,\dots T_\delta].$ Por \textit{going-down} existe $0\neq \p/\p_i\subset\p_{i+1}/\p_i$ ideal primo de $A/\p_i$ que se contrae a $\p''/\p_i$, contradiciendo la maximalidad de (\ref{eq:cadenaprimos1}).
	
	Ahora es suficiente demostrar que toda cadena maximal de ideales primos en un anillo de polinomios $C:=K[Y_1,\dots,Y_d]$ tiene $d+1$ términos.
	
	Razonamos por inducción sobre $d.$ Si $d=0$, es obvio. Sea $d\ge 1$ de nuevo una cadena maximal de ideales primos en $C$ como en (\ref{eq:cadenaprimos1}), donde $\p_0=0.$ Sea $C'=K[Z_1,\dots,Z_d]\subset C$ una extensión entera tal que $\p_1\cap K[Z_1,\dots, Z_d]=\langle Z_{\delta+1},\dots, Z_d\rangle=:\p_1'.$
	
	Tenemos que $\p_1\neq 0$, $\delta\le d-1$,¡ y $\hit(\p_1)=1$ (por ser (\ref{eq:cadenaprimos1}) maximal). Además, se verifican las hipótesis de \textit{going-down} entre $C$ y $C'.$ Así, $\hit{\p_1'}=1$ y no puede ser $\delta+1\le d.$ Así, $\delta=d-1$ y $C_1':=K[Z_1,\dots, Z_{d-1}]\cong C'/\p_1'\subset C/\p_1=:C_1$ es una normalización. Como (\ref{eq:cadenaprimos1}) proporciona una cadena maximal en $C_1$ (por el teorema de la correspondencia), $\p_m/\p_1\supsetneq\cdots\supsetneq\p_1/\p_1$ y, por (\ref{eq:cadenaprimos1}) la primera parte de su traza en $C_1'$ es también una cadena maximal. $C_1'$ es un anillo de polinomios y por inducción será $d-1\le m-1.$
\end{proof}

Esto que hemos visto tiene consecuencias geométricas. Sea $K$ un cuerpo algebraicamente cerrado y $V\subset \A^n(K)$ un conjunto algebraico irreucible de dimensión $d$, (i.e. $\mathcal{I}(V)\subset K[X_1,\dots,X_n]$ es un ideal primo y $\dimk{K[X_1,\dots,X_n]/\mathcal{I}(I)}=d$), entonces las cadenas maximales de subconjuntos algebraicos irreducibles propios de $V$ (i.e., cerrados irreducibles en la topología de Zariski contenidos estrictamente en $V$) tienen longitud $d$ y cualquier cadena estricta de subconjuntos algebraicos irreducibles de $V$ se puede refinar a una maximal. Esto último significa que las dimensiones disminuyen en uno por cada paso.


\end{document}

\end{document}
