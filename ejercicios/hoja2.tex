\documentclass[./ejercicios.tex]{subfiles}
\begin{document}
\paragraph{Ejercicio 1} \textit{ Sea $A$ un anillo, $\mathfrak a$ un ideal de $A$, y $M$ un $A$-módulo. Probar que $A/\mathfrak a \otimes M \cong M/\mathfrak a M $.}

Consideramos la cadena exacta $0\longrightarrow \mathfrak a \longrightarrow A \longrightarrow A/\mathfrak a \longrightarrow 0$ y la tensorizamos por $M$ tal que

\[ \mathfrak a \otimes M \longrightarrow A\otimes M \longrightarrow A/\mathfrak a \otimes M \longrightarrow 0 \]

que sabemos que es exacta. Por tanto, $\pi \otimes 1_M: A \otimes M \to A/\mathfrak a \otimes M $ es sobreyectiva, y aplicando el primer teorema de isomorfía $A \otimes M / \ker (\pi \otimes 1_M) \cong A/\mathfrak a \otimes M $. Por ser exacta, el núcleo coincide con la imagen de $i \otimes 1_M$, que es $\mathfrak a \otimes M$. Además, $A\otimes M \cong M$ vía el isomorfismo $a\otimes m \to am$, y la imagen de $\mathfrak a \otimes M \subset A\otimes M$ por esta aplicación es $\mathfrak a M$, lo que concluye la demostración.

\paragraph{Ejercicio 2} \textit{Sean $M, N \in \Mod_A$ y $\phi:M\to N$ y $\psi:N\to M$ homomorfismos tales que $\phi = \phi \circ \psi \circ \psi$ y $\psi = \psi \circ \phi \circ \psi$. Demostrar que:}

\begin{enumerate}
  \item $\im(\phi) = \im (\phi \circ \psi)$ y $\ker (\phi) = \ker(\psi \circ \phi)$.
  \item $M= \ker(\phi)\oplus \im(\psi)$.
\end{enumerate}

Definimos las aplicaciones: $f_1:M\to \ker \phi$ dada por $f_2:x\mapsto x-\psi \circ \phi (x)$ y $M\to \im \psi$ dada por $x\mapsto \psi \circ \phi (x)$. La segunda es claro que está bien definida, y la primera se comprueba que $\phi(x-\psi\circ \phi(x)) = \phi(x)-\phi\circ \psi \circ \phi(x) = \phi(x)-\phi(x) = 0$.

Tomamos $F=(f_1,f_2)$ y vemos que es nuestro isomorfismo. Es inyectiva porque si $(0,0) = (x-\psi \circ \phi (x), \psi \circ \phi (x))$ entonces $\psi \circ \phi (x) = 0$ y por tanto la primera coordenada dice $x=0$. Por otra parte, dado $(x,y)\in \ker \phi \oplus \im \psi$ definimos $m=x+y \in M$ y observamos que como $y\in \im \psi$ existe $z\in N$ con $y=\psi(z)$, y entonces: $f_2(m) =  \psi \circ \phi (y) = \psi \circ \phi \circ \psi(z) = \psi (z) = y $, y por tanto $f_1(m) = m - f_2(m) = (x+y)-y = x$.

\paragraph{Ejercicio 3}\textit{Sea $M\in \Mod_A$, $M^* = \Hom_A(M, A)$. Demostrar que la aplicación $\Phi:M\to M^{**}$ dada por $m \to \operatorname{eval}_m$ es un homomorfismo de $A$-módulos. Poner un ejemplo en que no es isomorfismo. Demostrar que, si $M$ es finitamente generado y proyectivo, entonces sí es isomorfismo.}

Se cumple que para cualesquiera $m, n \in M$, $a, b \in A$, $f\in M*$

\begin{equation}
  \Phi(a m+ bn) (f) = f(am+bn) = af(m)+bf(n) = a\Phi(m)(f)+b\Phi (n)(f)
\end{equation}
usando la $A$-linealidad de $f$.

Observamos que si $M$ es finitamente generado, entonces $M^*$ es finitamente generado y, por recurrencia, $M^{**}$ también es finitamente generado. Efectivamente, si $\set{m_j}_{j=1}^n$ es el conjunto de generadores de $M$, entonces las funciones $\varphi_i:M\to A$ dadas por $\varphi_i(m_j)= 1_A \delta_{ij}$ son generadores de $M^*$, ya que para toda $f:M \to A$ y para todo $x=\sum_{j=1}^n\lambda_jm_j\in M$ tenemos que, si $\mu_j = f(m_j)$, entonces
\[f(x) = \sum_{j=1}^n \lambda_j f(m_j) = \sum_{j=1}^n \varphi_j(x) \mu_j  \]
es decir, $f = \sum_{j=1}^n \mu_j \varphi_j$.

\paragraph{Ejercicio 7} \textit{Sea $m,n \in Z^+$. Demostrar que $\Z / \langle n \rangle \otimes \Z / \langle m \rangle \cong \Z/ \langle d \rangle$ donde $d = \gcd(m,n)$.}

Se puede hacer escribiendo una aplicación bilineal del producto cartesiano en $\Z/\langle d \rangle$, tensorizando, y después encontrando la inversa.

Usamos el ejercicio 1 con $A= \mathbb{Z}, \mathfrak{a} = \langle n \rangle, M = \mathbb{Z} / \langle m \rangle$. Entonces

\begin{equation}
  \mathbb{Z}/\langle n\rangle \otimes \mathbb{Z} / \langle m \rangle \cong \frac{\mathbb{Z} / \langle m \rangle }{\langle n \rangle  \mathbb{Z} / \langle m \rangle}
\end{equation}

Demostramos la igualdad de $\Z$-módulos

\begin{equation}
  \langle n \rangle  (\mathbb{Z}/ \langle m \rangle) = \langle \bar n \rangle = (n+\langle m  \rangle) ( \mathbb Z / \langle m\rangle )
\end{equation}

Un elemento de $\langle n \rangle  (\mathbb{Z}/ \langle m \rangle) $ es de la forma suma finita

\begin{equation}
\sum_i (a_i n ) (x_i + \langle m \rangle)  = n(  \sum_i a_i x_i) + \langle m \rangle  = (n+ \langle m \rangle)  ( \sum_i a_i x_i + \langle m \rangle) \in \langle \overline n \rangle
\end{equation}

y el otro contenido es automático.

Queremos aplicar el teorema de la correspondencia. Buscamos escribir el ideal $\langle \overline{n} \rangle $ del anillo ${ \mathbb Z}/ \langle m \rangle$ con un numerador que sea un ideal de ${ \mathbb Z}$ que contiene al ideal del denominador. Este es $\langle n,m \rangle / \langle m \rangle$ y sabemos que $\langle n,m \rangle = \langle d \rangle$ donde $d= \gcd(n,m)$. Por tanto

\begin{equation}
  \mathbb{Z}/\langle n\rangle \otimes \mathbb{Z} / \langle m \rangle \cong \frac{\mathbb{Z} / \langle m \rangle }{ \langle d \rangle / \langle m\rangle } \cong \Z / \langle d \rangle
\end{equation}

\paragraph{Ejercicio 5}
\subparagraph{i)} Sean $\set{k_i}\subset K$ y $\set{a_i}\subset A$, $i\in\set{1,\dots,r}$ tales que
$$\sum_{i}a_ik_i=0_K.$$
De esta igualdad se sigue también $\sum_{i}a_ik_i=0_M$ entendiendo los elementos $k_i$ como elementos de $M$ mediante la inclusión. Dado que $M$ es plano, existen $\set{m_j'\in M}$ y $\lambda_{ij}\in A$, $j\in\set{1,\dots, s}$, tales que $k_i=\sum_j\lambda_{ij}m_j'$ y $\sum_i a_i\lambda_{ij}$ para cada $i$ y $j$.

De las anteriores igualdades se sigue que $0_N=\sum_j\lambda_{ij}[m_j']$ para cada $i$, es decir, existen $[m(i)_{l}'']\in N$ y $\mu(i)_{jl}\in A$, donde $l\in\set{(i,1),\dots,(i,n(i))}=:J(i)$, tales que $m_j'=\sum_l\mu(i)_{jl}m(i)_l''$ y $\sum_{j}\lambda_{ij}\mu(i)_{jl}=0_A.$ Así, existen $\set{k(i)_j'}\in K$ tales que
$$m_j'=k(i)_j'+\sum_l\mu(i)_{jl}m(i)_l''\Longleftrightarrow k(i)_j'=m_{j}'-\sum_{l}\mu(i)_{jl}m(i)_l''.$$

Definimos ahora, para cada $t\in\set{1,\dots,rs}$ los elementos
$$k_t''=k(c(t))'_{r(t)},$$
donde $t=c(t)s+r(t),$ y
$$\gamma_{it}:=\left\{\begin{array}{cc}
\lambda_{i\ r(t)}&\text{si}\ c(t)=i\\
0&\text{si}\ c(t)\neq i
\end{array}\right..$$

De esta forma se tiene, fijada $i$,
\begin{align*}
\sum_t\gamma_{it}k_t''=\sum_j\lambda_{ij}k(i)_{j}''&=\sum_j\lambda_{ij}m_j'-\sum_j\lambda_{ij}\left(\sum_l\mu(i)_{jl}m_{ij}''\right)\\
&=\sum_j\lambda_{ij}m_j'-\sum_l\left(\sum_j\lambda_{ij}\mu(i)_{jl}\right)m_{ij}''=k_i
\end{align*}

y, fijada $t$,
$$\sum_i a_i\gamma_{it}=\sum_i a_i\lambda_{i\ r(t)}=\sum_i a_i\lambda_{ij}=0_A,$$
donde $j=r(t)$. De esta forma, $K$ es plano.

\subparagraph{ii)} Sean $\set{m_i}\subset M$ y $\set{a_i}\subset A$, $i\in\set{1,\dots,r}$ tales que
$$\sum_{i}a_im_i=0_M.$$
En particular, proyectando al cociente se tiene
$$\sum_{i}a_i[m_i]=0_N$$
y por ser plano existen $[m_j']\in N$ y $\lambda_{ij}\in A$, $j\in{1,\dots,s}$, de forma que $[m_i]=\sum_{j}\lambda_{ij}[m_j']=\left[\sum_{j}\lambda_{ij}m_j'\right]$ y $\sum_ia_i\lambda_{ij}=0_N$. De esta forma, existen $\set{k_i}\subset K$ tales que
$$m_i=k_i+\sum_{j}\lambda_{ij}m_j'.$$
Considerando de nuevo la suma inicial resulta
$$\sum_i a_im_i=\sum_i\left(k_i+\sum_{j}\lambda_{ij}m_j'\right)=\sum_i a_ik_i +\sum_j\left(\sum_ia_i\lambda_{ij}\right)m_j'=\sum_i a_ik_i,$$
es decir, $\sum_i a_ik_i=0_M$. Como $\sum_i a_ik_i\in K$, también $\sum_i a_ik_i\in K=0_K$ y existen $k_l'\in K$ y $\mu_{il}\in A$ tales que $k_i=\sum_l\mu_{il}k_l'$ y $\sum_i a_i\mu_{il}=0_A.$

Para concluir basta definir los siguientes elementos:
$$m_t'':=\left\{\begin{array}{cc}
m_{t}'&t\in\set{1,\dots,j}  \\
k_{t-j}'&t\in\set{j+1,\dots,j+l}
\end{array}\right.$$
y
$$\gamma_{it}:=\left\{\begin{array}{cc}
\lambda_{it}&t\in\set{1,\dots,j}  \\
\mu_{i\ t-j}&t\in\set{j+1,\dots,j+l}
\end{array}\right..$$
Tenemos así, fijada $i$,
$$\sum_t\gamma_{it}m_t''=\sum_{j}\lambda_{ij}m_j'+\sum_l\mu_{il}k_l'=\sum_{j}\lambda_{ij}m_j'+k_i= m_i$$
y, fijada $t$ ($t\in\set{1,\dots,j}$ o $t\in\set{j+1,\dots,j+l}$),
$$\sum_i a_i\gamma_{it}=0_A$$

\paragraph{Ejercicio 8} Sea $M$ un $A$-módulo proyectivo. Sabemos que existen $I$ conjunto de índices y $K$ $A$-módulo tales que
$$A^{(I)}\overset{\varphi}{\cong} K\oplus M.$$
Para cada $a\in A^{(I)}$, denotamos a su imagen por $\varphi$ como $(\varphi(a)_K,\varphi(a)_M)$.

Ahora, dada una sucesión exacta
$$0\longrightarrow N'\overset{f}{\longrightarrow}N$$
se tiene que la sucesión
$$0\longrightarrow A^{(I)}\otimes N'\overset{\operatorname{Id}_{A^{(I)}}\otimes f}{\longrightarrow} A^{(I)}\otimes N$$
también lo es.

Es claro que $\Phi:=\varphi\otimes \operatorname{Id}_{N'}: A^{(I)}\otimes N'\longrightarrow (K\otimes N')\oplus(M\otimes N')$, $\Psi:=\varphi\otimes \operatorname{Id}_{N}A^{(I)}\otimes N\longrightarrow (K\otimes N)\oplus(M\otimes N)$ son dos isomorfismos. Consideremos la sucesión
\begin{equation}\label{Hoja2ej8}
0\longrightarrow (K\otimes N')\oplus(M\otimes N')\overset{(\operatorname{Id}_{K}\otimes f)\oplus (\operatorname{Id}_{M}\otimes f)}{\longrightarrow} (K\otimes N)\oplus(M\otimes N)
\end{equation}
y veamos que se da la igualdad
$$\Psi^{-1}\circ(\operatorname{Id}_{K}\otimes f)\oplus (\operatorname{Id}_{M}\otimes f)\circ\Phi=\operatorname{Id}_{A^{(I)}}\otimes f,$$
es decir, que los diagramas conmutan: dado $(a,n')\in A^{(I)}\otimes N'$ arbitrario, se tiene
\begin{align*}
[\Psi^{-1}\circ&(\operatorname{Id}_{K}\otimes f)\oplus (\operatorname{Id}_{M}\otimes f)\circ\Phi]\ (a\otimes n')=\\
&=[\Psi^{-1}\circ(\operatorname{Id}_{K}\otimes f)\oplus (\operatorname{Id}_{M}\otimes f)](\varphi(a)_K\otimes n',\varphi(a)_M\otimes n')=\\
&=\Psi^{-1}(\varphi(a)_K\otimes f(n'),\varphi(a)_M\otimes f(n'))=(a,f(n'))=\operatorname{Id}_{A^{(I)}}\otimes f(a,n').
\end{align*}
Por esto que acabamos de ver, la sucesión ($\ref{Hoja2ej8}$) es exacta, es decir, la aplicación $F:=(\operatorname{Id}_{K}\otimes f)\oplus (\operatorname{Id}_{M}\otimes f)$ es inyectiva.
Así, para cada $m\otimes n'\in M\otimes n$, si $F(m\otimes n')=\operatorname{Id}_{M}\otimes f(m\otimes n')=0,$ entonces $m\otimes n'=0_{(K\otimes N')\oplus(M\otimes N')}$ y $m\otimes n'=0_{(M\otimes N')}$; es decir, $\operatorname{Id}_{M}\otimes f$ es inyectiva y $M$ plano.

\paragraph{Ejercicio 10}
\subparagraph{i)} Procedamos como indica el enunciado por inducción sobre $n$.

Para $k=1$, por ser $M$ un submódulo de $A$, tenemos que
\begin{itemize}
	\item para cualesquiera $m_1,m_2\in M$ se verifica $m_1-m_2\in M$ y,
	\item dados $m\in M$ y $a\in A$, $am\in M;$
\end{itemize}
es decir, $M$ es un ideal de $A$. Por ser $A$ DIP existe $m\in A$ tal que $M=\langle m\rangle$. Más aún, ser DIP implica ser DFU y, por esto, para todo $x\in M$ existe $a\in A$ tal que $x=am$ y esta expresión es única. Así, tenemos que $\{m\}$ es base de $M$ y por la caracterización de los módulos libres, $M$ lo es: concretamente, $M\cong A$.

Supongamos cierto el resultado para toda $k<n$ y veámoslo para $k=n$. En primer lugar, la sucesión
$$0\longrightarrow A^{(n-1)}\longrightarrow A^{(n)}\longrightarrow A\longrightarrow 0$$
es escindida en virtud de que $A^{(n)}\cong A^{(n)}\oplus A$ y de la caracterización de las sucesiones escindidas.

Por otra parte, supongamos que no existe $k<n$ tal que $M\subseteq A^{(n-1)}$ y veamos que se tienen los isomorfismos $M':=A^{(n-1)}\cap M\cong A^{(n-1)}$ y $M'':=M/{A^{(n-1)}\cap M}\cong A$.
Comencemos por el segundo de ellos. La suposición nos asegura que $M/{A^{(n-1)}}\neq \{0_M\}$ y concretamente $M/{A^{(n-1)}}\subseteq A^{(n)}/{A^{(n-1)}}\cong A$ y $M''\overset{\Phi}{\cong}A$ por el caso base.
Sean $\Phi^{-1}=:\overline{m},m\in M$ y consideremos $[m]\in M''$. Existe un único $a_m\in A$ tal que $m=a_m\overline{m}$. Así, podemos definir
$$\begin{array}{rrcl}
\sigma:&M''&\longrightarrow&M\\
&[m]&\longrightarrow&a_m\overline{m}
\end{array},$$
de forma que $\sigma\circ g=\operatorname{Id}_M,$ donde
\begin{equation}\label{equation:Hoja2ej10}
0\overset{}{\longrightarrow}M'\overset{f}{\longrightarrow}M\overset{g}{\longrightarrow}M''.
\end{equation}
De esto resulta que ($\ref{equation:Hoja2ej10}$) es escindida y $M\cong M'\oplus M''$.

Sigamos con el otro isomorfismo. Es claro que $M'$ es submódulo de $A^{(n-1)}$, de esta forma la hipótesis de inducción nos dice que $M'\cong A^{(s)}$ para cierta $s\le n-1$. Ahora bien, necesariamente $s=n-1$ pues en caso contrario, por lo que acabamos de ver, $M\cong A^{s}\oplus A$, que es absurdo.

Con todo,
$$M\cong M'\oplus M''\cong A^{(n-1)}\oplus A\cong A^{(n)}.$$

\subparagraph{ii)} Estas comprobaciones son rutinarias.

En primer lugar, dados $m_1,m_2\in T(M)$ existen $a_1,a_2\in A$ tales que $a_im_i=0_M$. Así, $a_1a_2(m_1-m_2)=0_M$ y $m_1-m_2\in T(M).$ Por otra parte, para todo $a\in A$ se tiene que $a_1(am_1)=a(a_1m_1)=0_M,$ es decir, $am_1\in T(M)$. Vemos así que $T(M)$ es submódulo de $M$.

Ahora, la igualdad
$$T(M)=\bigcup_{a\in A\setminus\{0_A\}}\ker(\mu_a)$$ nos dice que $T(M)=\{0_M\}$ si, y sólo si, $\mu_a$ es inyectiva para toda $a\in A\setminus\{0_A\}.$

Por último, dados $[m]\in M/T(M)$ y $a\in A$
$$a[m]=[0_M]\Longleftrightarrow [am]=[0_M]\Longleftrightarrow [m]=[0_M].$$

\subparagraph{iii)} Veamos que $\{x_1,\dots,x_r\}$ es base de $N$. En primer lugar, por la propia definición de $N$, para cada $n\in N$ existen $\{\lambda_i\}_i$ tales que
\begin{equation}\label{equation:Hoja2ej102}
n=\sum_{i=1}^r\lambda_ix_i.
\end{equation}
Veamos que son únicos. Supongamos que el conjunto $\{\lambda_i'\}_i$ también verifica ($\ref{equation:Hoja2ej102}$), por ser así
$$\sum_{i=1}^r(\lambda_i-\lambda_i')x_i.$$
Dado que $N$ es un $A$-módulo sin torsión $(\lambda_i-\lambda_i')x_i\neq 0_N$ para toda $i$, además, como $\{x_1,\dots,x_r\}$ es un sistema de generadores linealmente independiente tenemos $\lambda_i=\lambda_i'$ para toda $i$.
De esta forma, $N$ es un $A$-módulo libre de rango $r$.

Ahora, por la condición de maximalidad del sistema de generadores de $N$, tenemos que para cada $i\in\{r+1,\dots,s\}$ existen $\lambda_i\neq 0_A$ y $\{\alpha_j\}_{j=1}^r\subset A$ tales que
$$\lambda_ix_i=\sum_{j=1}^r\alpha_jx_j.$$
Definamos los elementos $\gamma:=\prod_{j=r+1}^s\lambda_j$ y $\gamma_i:=\prod_{j\in\set{r+1,\dots,s}\setminus\{i\}}\lambda_j$ y sea $x\in M$. Como $M=\langle x_1,\dots, x_s\rangle$, existen $\{\mu_i\}_{i=1}^s\subset A$ tales que $x=\sum_i\mu_ix_i$ y resulta
$$\gamma x=\sum_{i=1}^s\gamma\mu_ix_i=\sum_{i=1}^r\gamma\mu_ix_i+\sum_{i=r+1}^s\lambda\mu_ix_i=\sum_{i=1}^r\gamma\mu_ix_i+\sum_{i=r+1}^s(\gamma_i\mu_i)(\lambda_ix_i),$$
es decir, $\gamma x\in N$. Así, $\gamma M\subset N$.

\subparagraph{iv)} Para concluir el ejercicio, hagamos uso de los apartados anteriores. Sea $M$ un $A$-módulo sin torsión finitamente generado, digamos por los elementos $\{x_1,\dots,x_s\}$. Sabemos que existen un submódulo de $M$, $N$, finitamente generado y $A$-módulo libre y un elemento $a\in A$ tal que $aM\subset N$. Así, podemos considerar $aM$ un submódulo de $A^{(r)}$, donde $r$ es el rango de $N$; es decir, $aM$ es $A$-módulo libre. Por último, dado que $M$ está libre de torsión, la aplicación $\mu_a$ (definida en ii)) es un isomorfismo sobre $\operatorname{im}(\mu_a)\subset aM$ y podemos considerar de nuevo $M$ como submódulo de $A^{(s)}$ para cierta $s\in\N$. Con esto concluimos que $M$ es $A$-módulo libre.

\paragraph{Ejercicio 14} Por definición de conjunto multiplicativamente cerrado tenemos tanto que $0_A\neq S$ como que $1_A\in S$. De lo primero se deduce que $s^n\neq 0_A$ para toda $n\in\N$ y, de lo segundo, que $s$ no es divisor de cero. Lo primero es inmediato, veamos lo segundo.

En primer lugar, si existe $k\in\N$ tal que existe $a\in A$ de forma que $s^ka=0_A$, entonces basta tomar $a':=s^{k-1}a$ y se tiene que $sa'=0_A$. Así, $s$ no es divisor de cero si, y sólo si, toda potencia suya verifica no ser divisor de cero. Por otra parte, tenemos que existe $n_0$ (que podemos considerar mínima) de forma que $s^{n_0}=1_A$. Supongamos que $sa=0_A$ para cierto $a\in A$, por ser así, se tiene $0_A=s^{n_0-1}(sa)=s^{n_0}a=a$ y $s$ no es divisor de cero.

Estamos ya en condiciones de demostrar lo que se nos pide. Definamos el siguiente homomorfismo
$$\begin{array}{rrcl}
i:&A[T]&\longrightarrow&S^{-1}A[T]\\
&\operatorname{p}(T):=\sum_{i=1}^ra_iT^i&\longrightarrow&\overline{\operatorname{p}}(T):=\sum_{i=1}^r\delta_S(a_i)T^i
\end{array}.$$
Dado que $S\cap\operatorname{Div}_0A=\varnothing$, $\delta_S$ es inyectiva y así lo es también $i$. Ahora, considerando el homomorfismo evaluación en $\frac{1}{s}$,
$$\begin{array}{rrcl}
\operatorname{ev}_{\frac{1}{s}}:&S^{-1}A[T]&\longrightarrow&S^{-1}A\\
&\overline{\operatorname{p}}(T)&\longrightarrow&\overline{\operatorname{p}}\left(\frac{1}{s}\right)
\end{array}$$
tenemos el homomorfismo $\operatorname{ev}_{\frac{1}{s}}\circ\  i:A[T]\longrightarrow S^{-1}A.$

Veamos que $\ker(\operatorname{ev}_{\frac{1}{s}}\circ\  i)=\langle sT-1\rangle.$ La inclusión $\langle sT-1\rangle\subset \ker(\operatorname{ev}_{\frac{1}{s}}\circ\  i)$ es obvia, comprobemos la recíproca.

Sea $i(\operatorname{p})(T):=\sum_{i=0}^{r+1}\delta_S(p_i)T^i\in S^{-1}A[T]$, donde $\operatorname{p}\in A[T]$ y $r\in\N$ y tal que $i(\operatorname{p})(\frac{1}{s})=0_{S^{-1}A}.$ Sea también $\operatorname{h}(T):=\sum_{i=0}^r\frac{a_i}{s^{n_i}}T^i$ verificando
$$(\delta(s)T-\delta(1))\operatorname{h}(T)=\operatorname{p}(T).$$
Veamos que para cada $\set{1,\dots, r}$ se tiene $\frac{a_i}{s^{n_i}}=\frac{a^*_i}{1_A}$ para ciertas $a^*_i\in A$. Para ello, realizamos la multiplicación e igualamos coeficientes.
\begin{equation}
\begin{split}
(\delta(s)T-\delta(1))\operatorname{h}(T)&=\sum_{i=0}^{r}\frac{a_i}{s^{n_i-1}}T^{i+1}-\sum_{i=0}^r\frac{a_i}{s^{n_i}}T^i=\sum_{i=1}^{r+1}\frac{a_{i-1}}{s^{n_{i-1}-1}}T^{i}-\sum_{i=0}^r\frac{a_i}{s^{n_i}}T^i\\
&=\frac{a_r}{s^{n_r-1}}T^{r+1}+\sum_{i=1}^r\frac{s^{n_i}a_{i-1}-s^{n_{i-1}-1}a_{i}}{s^{n_{i-1}-1}s^{n_i}}T^{i}+\frac{a_0}{s^{n_0}}.
\end{split}
\end{equation}
Así, surgen las ecuaciones
\begin{equation}\label{equation:Hoja10ej14}
\begin{split}
a_r&=s^{n_r-1}p_{r+1}\\
s^{n_i}a_{i-1}-s^{n_{i-1}-1}a_{i}&=s^{n_{i-1}-1}s^{n_i}p_i\\
a_0&=s_0p_0
\end{split}
\end{equation}
y se desprende de la del medio que
$$s^{n_{i-1}-1}a_i=s^{n_i}a_{i-1}-s^{n_{i-1}-1}s^{n_i}p_i.$$
Probemos por inducción sobre $i$ que
$$a_i=s^{n_i}(s^ip_0-\sum_{k=1}^is^{i-k}p_k).$$
El caso base es obvio atendiendo a ($\ref{equation:Hoja10ej14}$). Si ahora lo suponemos para $t=i-1$, vemos que se cumple para $t=i$:
\begin{equation}
\begin{split}
s^{n_{i-1}-1}a_i&=s^{n_i}a_{i-1}-s^{n_{i-1}-1}s^{n_i}p_i\\
s^{n_{i-1}-1}a_i&=s^{n_i}s^{n_{i-1}}(s^{i-1}p_0-\sum_{k=1}^{i-1}s^{i-1-k}p_k)-s^{n_{i-1}-1}s^{n_i}p_i\\
a_i&=s^{n_i}s(s^{i-1}p_0-\sum_{k=1}^{i-1}s^{i-1-k}p_k)-s^{n_i}p_i\\
a_i&=s^{n_i}(s^ip_0-\sum_{k=1}^is^{i-k}p_k).
\end{split}
\end{equation}
De esta forma, tenemos que para toda $i\in\set{1,\dots, r}$ se cumple
$$\frac{a_i}{s^{n_i}}=\frac{s^ip_0-\sum_{k=1}^is^{i-k}p_k}{1_A}$$
y, además, se desprende que $s|p_{r+1}.$

Con todo, definiendo $\widetilde{\operatorname{h}}(T):=\sum_{i=1}^r\left[s^ip_0-\sum_{k=1}^is^{i-k}p_k\right]T^i$ tenemos que $(sT-1)\widetilde{\operatorname{h}}(T)=\operatorname{p}(T)$ y $\operatorname{p}\in\langle sT-1\rangle,$ es decir, si $\operatorname{p}\in A[T]$ verifica $\operatorname{ev}_{\frac{1}{s}}\circ\ i(\operatorname{p})=0_{S^{-1}A},$ entonces $\operatorname{p}\in \langle sT-1\rangle$ y $\ker(\operatorname{ev}_{\frac{1}{s}}\circ\ i)\subset\langle sT-1\rangle.$

Podemos concluir así por el Primer Teorema de Isomorfía que $S^{-1}A\cong A[T]/\langle sT-1\rangle.$
%
% \paragraph{Ejercicio 18}
% \textit{(i)} Para todo $x,y \in S_0$ se tiene $xy\in S_0$. Si no, existe $x \in A$ tal que $0 = xyz = x(yz)$ con lo que $x\in \Div_0$, absurdo. Por otro lado, $0\not \in S_0$ obviamente, y $1 \in S_0$ porque apra todo $a\neq 0$ se tiene $1\cdot a = a \neq 0$.
%
% \textit{(ii)} La aplicación $A \to S_0^{-1}A$ es inyectiva. Sea $x,y \in A$ tales que $\frac{x}{1} = \frac{y}{1}$ , entonces existe $s\in S_0$
%
% \paragraph{Ejercicio 19}
% Supongamos que $S^{-1}A$ no es un DFU. Entonces existe un $\frac{x}{s}$ con $x,s\in A$ que no es una unidad ni es nulo tal que
% $$
% \frac{x}{s} = \frac{x_1}{s_1} \dots \frac{x_n}{s_n} = \frac{y_1}{t_1} \dots \frac{y_m}{t_m}
% $$
% donde los $\frac{x_i}{s_i}$ y los $\frac{y_j}{t_j}$ son irreducibles en $S^{-1}A$, y $\frac{x_1}{s_1}$ y $\frac{y_1}{t_1}$ no son asociados. En particular, $x \neq 0$ porque como $A$ es DI $\delta_S$ es inyectiva,  y no existe $y \in A$ tal que $xy \in S$.
%
% Esto quiere decir que para toda unidad del anillo de fracciones $\frac{z}{w}$
%
% De la igualdad $(1)$ y teniendo en cuenta que $A$ es DI y $0 \not \in S$ sacamos que
% $$
% \begin{cases}
% \prod_{i}x_i \prod_jt_j = \prod_j y_j \prod_is_i \\
% x\prod_is_i = s\prod_i x_i \\
% x \prod_jt_j = s\prod_j y_j
% \end{cases}
% $$

\paragraph{Ejercicio 17}
$(i)\Rightarrow (iii)$. Si $x = 0$ es trivial. Sea $x\neq 0$ tal que $x\in \mathfrak{p}_0$. Supongamos que para todo $y\in  A$, si $xy = 0$ entonces $y \in \mathfrak{p}_0$. Entonces $\operatorname{anul}_A(x) = \{y \in A \vert \, xy = 0_A\} \subsetneq \mathfrak{p}_0$ contra la minimalidad de $\mathfrak{p}_0$.
El contenido es estricto porque $x \not \in \operatorname{anul}_A(x)$ ya que en tal caso, $x \in \mathfrak{N}_A = \langle 0 \rangle$, pero $x \neq 0$.
Entonces ha de existir $y \in A$ tal que $xy = 0$ e $y\not \in \mathfrak{p}_0$.

$(iii)\Rightarrow (ii)$. Sea cualquier $\frac{x}{s}\in \mathfrak{p}_0^e$, entonces $x \in \mathfrak{p}_0$ y por hipótesis existe un $y \not \in \mathfrak{p}_0$ con $xy = 0$. Entonces
$$
\frac{ 0 }{ 1 } = \frac{xy}{s} = \frac{x}{s} \frac{y}{1}
$$
pero $\frac{y}{1}$ es una unidad en $A_{\mathfrak{p}_0}$ por tanto $\frac{0}{1} = \frac{ x }{ s }$ y así $\mathfrak{p}_0^e = \langle \frac{0}{1} \rangle$. Y como el único ideal maximal es el nulo, $A_{\mathfrak{p}_0}$  es un cuerpo.

$(ii)\Rightarrow (i)$. $A_{\mathfrak{p}_0}$  es un cuerpo si y solo si los únicos ideales son el nulo y el total. En particular, no hay ideales primos ``salvo el nulo''. Por la biyección existente entre los primos de  $A_{\mathfrak{p}_0}$  y los primos de $A$ contenidos en $\mathfrak{p}_0$, no hay ningún primo no nulo contenido en $\mathfrak{p}_0$ y así este es minimal.

\paragraph{Ejercicio 20}

Consideramos la aplicación
\begin{align*}
  \Phi: (A/\af)_{\mathfrak{p}/\af} &\longrightarrow A_\mathfrak{p}/\af^e\\
  \frac{x+\af}{s+\af} &\longmapsto \frac{x}{s}+\af^e
\end{align*}
Veamos que está bien definida.
$$
\frac{x}{s}+ \af^e = \frac{y}{t}+\af^e \iff \frac{x}{s}-  \frac{y}{t}\in \af^e  \iff \frac{xt-ys}{st} = \frac{a}{s}
$$
con $a\in \af, u \not \in \mathfrak{p}$. Lo anterior se cumple si y solo si existe $w \not \in \mathfrak{p}$ tal que
$$
w ( ast - u(xt-ys)) = 0_A \underset{v = wu}{\iff}  v(xt-ys) = (wst)a \in \mathfrak{a} \iff$$
$$
\iff v(xt-ys) + \af = 0+\af  \iff  \frac{x+\af}{s+\af}= \frac{y+\af}{t+\af}
$$
Esto prueba también la inyectividad de la aplicación. La sobreyectividad es obvia.

Por otra parte, la aplicación entre espectros la vemos como composición de dos biyecciones: la que da la extensión-contracción y la que da el teorema de la correspondencia
$$
\Spec((A/\af)_{\mathfrak{p}/\af}) 	\longleftrightarrow \left \{ \frac{\mathfrak{q}}{\af}\in \Spec(A/\af) \vert \frac{\mathfrak{q}}{\af} \subset \frac{\mathfrak{p}}{\af} \right\} \longleftrightarrow \left  \{ \mathfrak{q} \in \Spec(A) \vert \, \af \subset \mathfrak{q} \subset \mathfrak{p} \right\}
$$

\paragraph{Ejercicio 22}
Hay un isomorfismo natural $S^{-1}(M'\oplus M) \cong S^{-1}M' \oplus S^{-1}M''$
\begin{align*}
S^{-1}(M'\oplus M) &\longrightarrow S^{-1}M' \oplus S^{-1}M''\\
\frac{(m',m'')}{s} &\longmapsto  \left( \frac{m'}{s}, \frac{m''}{s} \right)
\end{align*}

cuya inversa es

\begin{align*}
S^{-1}M' \oplus S^{-1}M'' &\longrightarrow S^{-1}(M'\oplus M)\\
\left( \frac{m'}{s}, \frac{m''}{t} \right) &\longmapsto \frac{(tm',sm'')}{st}
\end{align*}

Están ambas bien definidas.

$$
\frac{(m',m'')}{s} = \frac{(n',n'')}{t} \iff r[t(m',m'')-s(n',n'')] = (0,0) \iff
$$
$$
\begin{cases}
  r(tm'-sn') = 0\\ r(tm''-sn'') = 0
\end{cases} \Rightarrow (\frac{m'}{s},\frac{m''}{s}) = (\frac{n'}{t},\frac{n''}{t} )
$$

Entonces si $\mathfrak{p}\in \operatorname{supp}(M)$
$$
M_\mathfrak{p} = 0 \iff (M'\oplus M)_\mathfrak{p} =\left  \{\frac{(0,0)}{1} \right\} \iff M'_\mathfrak{p} \oplus M''_\mathfrak{p} = \left \{\left (\frac{0}{1}, \frac{0}{1} \right) \right\}
$$

y así $\mathfrak{p}\in \operatorname{supp}(M') \cap \operatorname{supp}(M'')$.

\end{document}
